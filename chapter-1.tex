\chapter*{跋}

\section{江之兰跋}
抱异疾者多奇梦,梦所未到之境,梦所未见之事,以心为君主之官,邪干之故如此,此则病也,非梦也。至若梦木撑天,梦河无水,则休咎应之;梦牛尾梦蕉鹿,则得失应之。此则梦也,非病也。心斋之《幽梦影》,非病也,非梦也,影也。影者维何?石火之一敲,电光之一瞥也。东坡所谓一掉头时生老病,一弹指顷去来今也。昔人云芥子纳须弥,而心斋则于倏忽备古今也。此因其心闲手闲,故弄墨如此之闲适也。心斋盖长于勘梦者也,然而未可向痴人说也。



己巳三月既望 寓东淘香雪斋江之兰草


\section{葛元煦跋}
余习闻《幽梦影》一书,著墨不多,措词极隽,每以未获一读为恨事。客秋南沙顾耐圃茂才示以钞本,展玩之馀,爱不释手。所惜尚有残阙,不无馀憾。今从同里袁翔甫大令处见有刘君式亭所赠原刊之本,一无遗漏,且有同学诸君评语,尤足令人寻绎。间有未评数条,经大令一一补之,功媲娲皇,允称全璧。爰乞重付手民,冀可流传久远。大令欣然曰:诺。故略叙其巓末云。



光绪五年岁次已卯冬十月仁和葛元煦理斋氏识


\section{杨复古跋}
昔人著书,间附评语,若以评语参错书中,则《幽梦影》创格也。清言隽旨,前于后喁,令读者如入真长座中与诸客周旋,聆其謦欬,不禁色舞眉飞,洵翰墨中奇观也。书名曰“梦”日“影”盖取六如之义。饶广长舌,散天女花,心灯意蕊,一印印空,可以悟矣。



乙未夏日震泽杨复古识


\section{张惣识跋}
昔人云:梅花之影,妙于梅花。窃意影子何能妙于花?惟花妙则影亦妙,枝干扶疏,自尔天然生动。凡一切文字语言,总是才人影子,人妙则影自妙。此册一行一句,非名言即韵语,皆从胸次体验而出,故能发人警省。片玉碎金,俱可宝贵。幽人梦境,读者勿作影响观可矣。                           南村张惣识
