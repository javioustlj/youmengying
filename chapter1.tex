\chapter{卷上}

\begin{yulu}{读书宜分四时}
读经宜冬,其神专也;读史宜夏,其时久也;读诸子宜秋,其致别也;读诸集宜春,其机畅也。
\begin{comments}
曹秋岳曰:“可想见其南面百城 时。” \\
庞笔奴曰:“读《幽梦影》则春、夏、秋、冬,无时不宜。”
\end{comments}
\end{yulu}

\begin{yulu}{独读与共读}
经传宜独坐读;史鉴宜与友共读。
\begin{comments}
孙恺似曰:“深得此中真趣,固难为不知者道。” \\
王景州曰:“如无好友,即红友亦可。”
\end{comments}
\end{yulu}

\begin{yulu}{善恶论}
无善无恶是圣人〈如:“帝力何有于我”,“杀之而不怨,利之而不庸”;“以直报怨,以德报德”;“一介不与,一介不取”之类〉,善多恶少是贤者〈如:颜子“不贰过,有不善未尝不知”;子路,“人告有过则喜”之类〉,善少恶多是庸人,有恶无善是小人〈其偶为善处,亦必有所为〉,有善无恶是仙佛〈其所谓善,亦非吾儒之所谓善也〉。
\begin{comments}
黄九烟曰:“今人‘一介不与’者甚多,普天之下皆半边圣人也。‘利之不庸’者亦复不少。” \\
江含征曰:“先恶后善是回头人,先善后恶是两截人。” \\
殷日戒曰:“貌善而心恶者是奸人,亦当分别。” \\
冒青若曰:“昔人云:‘善可为而不可为。’唐解元诗云:‘善亦懒为何况恶!’当于有无多少中,更进一层。”
\end{comments}
\end{yulu}

\begin{yulu}{人物有一知己可无恨}
天下有一人知己,可以不恨。不独人也,物亦有之。如菊以渊明为知己,梅以和靖为知己,竹以子猷为知己,莲以濂溪为知己,桃以避秦人为知己,杏以董奉为知己,石以米颠为知己,荔枝以太真为知己,茶以卢仝、陆羽为知己,香草以灵均为知己,鲈以季鹰为知己,蕉以怀素为知己,瓜以邵平为知己,鸡以处宗为知己,鹅以右军为知己,鼓以祢衡为知己,琵琶以明妃为知己。一与之订,千秋不移。若松之于秦始、鹤之于卫懿,正所谓不可与作缘者也。
\begin{comments}
查二瞻曰:“此非松、鹤有求于秦始、卫懿,不幸为其所近,欲避之而不能耳。” \\
殷日戒曰:“二君究非知松、鹤者,然亦无损其为松、鹤。” \\
周星远曰:“鹤于卫懿,犹当感恩。至吕政五大夫之爵,直是唐突十八公耳。” \\
王名友曰:“松遇封,鹤乘轩,还是知己。世间尚有剧松煮鹤者,此又秦、卫之罪人也。” \\
张竹坡曰:“人中无知己,而下求于物,是物幸而人不幸矣。物不遇知己而滥用于人,是人快而物不快矣。可见知己之难,知其难方能知其乐。”
\end{comments}
\end{yulu}

\begin{yulu}{四“忧”者}
为月忧云,为书忧蠹,为花忧风雨,为才子佳人忧命薄。真是菩萨心肠。
\begin{comments}
余淡心曰:“洵如君言,亦安有乐时耶!” \\
孙松坪曰:“所谓“君子有终身之忧”者耶!” \\
黄交三曰:“‘为才子、佳人忧命薄’一语,真令人泪湿青衫。” \\
张竹坡曰:“第四忧,恐命薄者消受不起。” \\
江含徵曰:“我读此书时,不免为蟹忧雾。” \\
竹坡又曰:“江子此言,直是为自己忧蟹耳。” \\
尤悔庵曰:“杞人忧天,嫠妇忧国,无乃类是。”
\end{comments}
\end{yulu}

\begin{yulu}{人、物不可无侣}
花不可以无蝶,山不可以无泉,石不可以无苔,水不可以无藻,乔木不可以无藤萝,人不可以无癖。
\begin{comments}
黄石闾曰:“‘事到可传皆具癖’,正谓此耳。” \\
孙松坪曰:“和长舆却未许借口。”
\end{comments}
\end{yulu}

\begin{yulu}{十声}
春听鸟声,夏听蝉声,秋听虫声,冬听雪声。白昼听棋声,月下听箫声。山中听松风声,水际听欸乃声。方不虚生此耳。若恶少斥辱、悍妻诟谇,真不若耳聋也。
\begin{comments}
黄仙裳曰:“此诸种声颇易得,在人能领略耳。” \\
朱菊山曰:“山老所居,乃城市山林,故其言如此。若我辈日在广陵城市中,求一鸟声,不啻如凤凰之鸣,顾可易言耶!” \\
释中洲曰:“昔文殊选二十五位圆通,以普门耳根为第一。今心斋居士耳根不减普门。吾他日选圆通,自当以心斋为第一矣。” \\
张竹坡曰:“久客者,欲听儿辈读书声,了不可得。” \\
张迂庵曰:“可见对恶少悍妻,尚不若日与禽虫周旋也。”又曰:“读此,方知先生耳聋之妙。”
\end{comments}
\end{yulu}

\begin{yulu}{择友酌酒}
上元须酌豪友,端午须酌丽友,七夕须酌韵友,中秋须酌淡友,重九须酌逸友。
\begin{comments}
朱菊山曰:“我于诸友中,当何所属耶?” \\
王武徵曰:“君当在豪与韵之间耳。” \\
王名友曰:“维扬丽友多,豪友少,韵友更少。至于淡友、逸友,则削迹矣。” \\
张竹坡曰:“诸友易得,发心酌之者为难能耳。” \\
顾天石曰:“除夕须酌不得意之友。” \\
徐砚谷曰:“惟我则无时不可酌耳。” \\
尤谨庸曰:“上元酌灯,端午酌采丝,七夕酌双星,中秋酌月,重九酌菊,则吾友俱备矣。”
\end{comments}
\end{yulu}

\begin{yulu}{物类神仙}
鳞虫中金鱼,羽虫中紫燕,可云物类神仙。正如东方曼倩避世,金马门人不得而害之。
\begin{comments}
江含徵曰:金鱼之所以免汤镬者,以其色胜而味苦耳。昔人有以重价觅奇特者以馈邑侯,邑侯他日谓之日:“贤所赠花鱼,殊无味。”盖已烹之矣。世岂少削圆方竹杖者哉!
\end{comments}
\end{yulu}

\begin{yulu}{入世与出世}
入世须学东方曼倩,出世须学佛印了元。
\begin{comments}
江含征曰:“武帝高明喜杀,而曼倩能免于死者,亦全赖吃了长生酒耳。” \\
殷日戒曰:“曼倩诗有云:‘依隐玩世,诡时不逢。’此其所以免死也。” \\
石天外曰:“入得世,然后出得世。入世出世,打成一片,方有得心应手处。”
\end{comments}
\end{yulu}

\begin{yulu}{择人共景}
赏花宜对佳人,醉月宜对韵人,映雪宜对高人。
\begin{comments}
余淡心日:“花即佳人,月即韵人,雪即高人。既已赏花、醉月、映雪,即与对佳人、韵人、高人无异也。” \\
江含徵曰:“若对此君仍大嚼,世间那有扬州鹤!” \\
张竹坡日:“聚花、月、雪于一时,合佳、韵、高为一人,吾当不赏而心醉矣。”
\end{comments}
\end{yulu}

\begin{yulu}{读人如读书}
对渊博友,如读异书;对风雅友,如读名人诗文;对谨饬友,如读圣贤经传;对滑稽友,如阅传奇小说。
\begin{comments}
李圣许曰:“读这几种书,亦如对这几种友。” \\
张竹坡曰:“善于读书、取友之言。”
\end{comments}
\end{yulu}

\begin{yulu}{楷、行、草如人}
楷书须如文人,草书须如名将。行书介乎二者之间,如羊叔子缓带轻裘,正是佳处。
\begin{comments}
程桦老曰:“心斋不工书法,乃解作此语耶!” \\
张竹坡曰:“所以羲之必做右将军。”
\end{comments}
\end{yulu}

\begin{yulu}{入诗与入画}
人须求可入诗,物须求可入画。
\begin{comments}
龚半千曰:“物之不可入画者,猪也,阿堵物也,恶少年也。” \\
张竹坡曰:“诗亦求可见得人,画亦求可像个物。” \\
石天外曰:“人须求可入画,物须求可入诗,亦妙。”
\end{comments}
\end{yulu}

\begin{yulu}{少年与老成}
少年人须有老成之识见,老成人须有少年之襟怀。
\begin{comments}
江含征日:“今之钟鸣漏尽、白发盈头者,若多收几斛麦,便欲置侧室,岂非有少年襟怀耶!独是少年老成者少耳。” \\
张竹坡日:“十七八岁便有妾,亦居然少年老成。” \\
李若金曰:“老而腐板,定非豪杰。” \\
王司直日:“如此方不使岁月弄人。”
\end{comments}
\end{yulu}

\begin{yulu}{天之本怀与别调}
春者天之本怀,秋者天之别调。
\begin{comments}
石天外曰:“此是透彻性命关头语。” \\
袁江中曰:“得春气者,人之本怀;得秋气者,人之别调。” \\
尤悔庵日:“夏者,天之客气;冬者,天之素风。” \\
陆云士日:“和神当春,清节为秋,天在人中矣。”
\end{comments}
\end{yulu}


\begin{yulu}{翰墨棋酒}
昔人云:若无花、月、美人,不愿生此世界。予益一语云:若无翰、墨、棋、酒,不必定作人身。
\begin{comments}
殷日戒曰:“枉为人身,生在世界者,急宜猛省。” \\
顾天石曰:“海外诸国,决无翰、墨、棋、酒,即有,亦不与吾同,一般有人,何也?” \\
胡会来曰:“若无豪杰、文人,亦不须要此世界。”
\end{comments}
\end{yulu}


\begin{yulu}{六“愿为”者}
愿在木而为樗〈不才终其天年〉,愿在草而为蓍〈前知〉,愿在鸟而为鸥〈忘机〉,愿在兽而为廌〈触邪〉,愿在虫而为蝶〈花间栩栩〉,愿在鱼而为鲲〈逍遥游〉。
\begin{comments}
吴园次曰:“较之《闲情》一赋,所愿更自不同。” \\
郑破水曰:“我愿生生世世为顽石。” \\
尤悔庵曰:“第一大愿。”又曰:“愿在人而为梦”。 \\
尤慧珠曰:“我亦有大愿,愿在梦而为影。” \\
弟木山曰:“前四愿皆是相反。盖‘前知’则必多‘才’,‘忘机’则不能‘触邪’也。”
\end{comments}
\end{yulu}

\begin{yulu}{古今人必有其偶双}
黄九烟先生云:古今人必有其偶双,千古而无偶者,其惟盘古乎?予谓盘古亦未尝无偶,但我辈不及见耳。其人为谁,即此劫尽时最后一人是也。
\begin{comments}
孙松坪曰:“如此眼光,何啻出牛背上耶!” \\
洪秋士日:“偶亦不必定是两人,有三人为偶者,有四人为偶者,有五、六、七、八人为偶者,是又不可不知。”
\end{comments}
\end{yulu}

\begin{yulu}{夏为三馀}
古人以冬为三馀。予谓当以夏为三馀:晨起者,夜之馀;夜坐者,昼之馀;午睡者,应酬人事之馀。古人诗云:“我爱夏日长。”洵不诬也。
\begin{comments}
张竹坡曰:“眼前问冬夏皆有馀者,能几人乎?” \\
张迂庵曰:“此当是先生辛未年以前语。”
\end{comments}
\end{yulu}

\begin{yulu}{庄周梦蝶}
庄周梦为蝴蝶,庄周之幸也;蝴蝶梦为庄周,蝴蝶之不幸也。
\begin{comments}
黄九烟曰:“惟庄周乃能梦为蝴蝶,惟蝴蝶乃能梦为庄周耳。若世之扰扰红尘者,其能有此等梦乎!” \\
孙恺似曰:“君于梦之中又占其梦耶!” \\
江含征曰:“周之喜梦为蝴蝶者,以其入花深也。若梦甫酣而乍醒,则又如嗜酒者梦赴席而为妻惊醒,不得不痛加诟谇矣。” \\
张竹坡曰:“我何不幸而为蝴蝶之梦者?”
\end{comments}
\end{yulu}

\begin{yulu}{七“可以邀”者}
艺花可以邀蝶,累石可以邀云,栽松可以邀风,贮水可以邀萍,筑台可以邀月,种蕉可以邀雨,植柳可以邀蝉。
\begin{comments}
曹秋岳曰:“藏书可以邀友。” \\
崔莲峰曰:“酿酒可以邀我。” \\
尤艮斋曰:“安得此贤主人?” \\
尤慧珠曰:“贤主人非心斋而谁乎?” \\
倪永清曰:“选诗可以邀谤。” \\
陆云士曰:“积德可以邀天,力耕可以邀地,乃无意相邀。而若邀之者,与邀名邀利者迥异。” \\
庞天池曰:“不仁可以邀富。”
\end{comments}
\end{yulu}

\begin{yulu}{景、境、声}
景有言之极幽,而实萧索者,烟雨也;境有言之极雅,而实难堪者,贫病也;声有言之极韵,而实粗鄙者,卖花声也。
\begin{comments}
谢海翁曰:“物有言之极俗而实可爱者,阿堵物也。” \\
张竹坡曰:“我幸得极雅之境。”
\end{comments}
\end{yulu}

\begin{yulu}{富贵之源}
才子而富贵,定从福慧双修得来。
\begin{comments}
冒青若曰:“才子富贵难兼。若能运用富贵,纔是才子,纔是福、慧双修。世岂无才子而富贵者乎?徒自贪著,无济于人,仍是有福无慧。” \\
陈鹤山曰:“释氏云:‘修福不修慧,象身挂璎珞;修慧不修福,罗汉供应薄。’正以其难兼耳。山翁发为此论,直是夫子自道。” \\
江含征曰:“宁可拼一副菜园肚皮,不可有一副酒肉面孔。”
\end{comments}
\end{yulu}


\begin{yulu}{新月与缺月}
新月恨其易沉,缺月恨其迟上。
\begin{comments}
孔东塘曰:“我唯以月之迟早为睡之迟早耳。” \\
孙松坪曰:“第勿使浮云点缀,尘滓太清,足矣。” \\
冒青若曰:“天道忌盈。沉与迟,请君勿恨。” \\
张竹坡曰:“易沉、迟上,可以卜君子之进退。”
\end{comments}
\end{yulu}

\begin{yulu}{灌园与薙草}
躬耕吾所不能,学灌园而已矣;樵薪吾所不能,学薙草而已矣。
\begin{comments}
汪扶晨曰:“不为老农,而为老圃,可云半个樊迟。” \\
释菌人曰:“以灌园、薙草自任,自待可谓不薄。然笔端隐隐有‘非其种者,锄而去之’之意。” \\
王司直曰:“予自名为识字农夫,得毋妄甚!”
\end{comments}
\end{yulu}

\begin{yulu}{十“恨”者}
一恨书囊易蛀,二恨夏夜有蚊,三恨月台易漏,四恨菊叶多焦,五恨松多大蚁,六恨竹多落叶,七恨桂荷易谢,八恨薜萝藏虺,九恨架花生刺,十恨河豚多毒。
\begin{comments}
江药庵曰:“黄山松幷无大蚁,可以不恨。” \\
张竹坡曰:“安得诸恨物尽有黄山乎!” \\
石天外曰:“予另有二恨:一曰‘才人无行’,二曰‘佳人薄命’。”
\end{comments}
\end{yulu}

\begin{yulu}{五“看”者}
楼上看山,城头看雪,灯前看月,舟中看霞,月下看美人,另是一番情境。
\begin{comments}
江允凝曰:“黄山看云,更佳。” \\
倪永清曰:“做官时看进士,分金处看文人。” \\
毕右万曰:“予每于雨后看柳,觉尘襟俱涤。” \\
尤谨庸曰:“山上看雪,雪中看花,花中看美人,亦可。”
\end{comments}
\end{yulu}

\begin{yulu}{摄魂倒思之景致}
山之光,水之声,月之色,花之香,文人之韵致,美人之姿态,皆无可名状,无可执著。真足以摄召魂梦,颠倒情思!
\begin{comments}
吴街南曰:“以极有韵致之文人,与极有姿态之美人,共坐于山、水、花、月间,不知此时魂梦何如?情思何如?”
\end{comments}
\end{yulu}

\begin{yulu}{自主梦魂}
假使梦能自主,虽千里无难命驾,可不羡长房之缩地;死者可以晤对,可不需少君之招魂;五岳可以卧游,可不俟婚嫁之尽毕。
\begin{comments}
黄九烟曰:“予尝谓‘鬼有时胜于人’,正以其能自主耳。” \\
江含徵曰:“吾恐‘上穷碧落下黄泉,两地茫茫皆不见’也。” \\
张竹坡曰:“梦魂能自主,则可一生死,通人鬼。真见道之言矣。”
\end{comments}
\end{yulu}

\begin{yulu}{不幸而非缺}
昭君以和亲而显,刘蕡以下第而传,可谓之不幸,不可谓之缺陷。
\begin{comments}
江含徵曰:“若故折黄雀腿而后医之,亦不可。” \\
尤悔庵日:“不然,一老宫人,一低进士耳。”
\end{comments}
\end{yulu}

\begin{yulu}{爱花与爱美人}
以爱花之心爱美人,则领略自饶别趣;以爱美人之心爱花,则护惜倍有深情。
\begin{comments}
冒辟疆曰:“能如此,方是真领略、真护惜也。” \\
张竹坡曰:“花与美人,何幸遇此东君。”
\end{comments}
\end{yulu}

\begin{yulu}{舍生香取解语者}
美人之胜于花者,解语也;花之胜于美人者,生香也。二者不可得兼,舍生香而取解语者也。
\begin{comments}
王勿翦曰:“飞燕吹气若兰,合德体自生香,薛瑶英肌肉皆香。则美人又何尝不生香也!”
\end{comments}
\end{yulu}

\begin{yulu}{窗外观纸窗作字}
窗内人于纸窗上作字,吾于窗外观之,极佳。
\begin{comments}
江含徵曰:“若索债人于窗外纸上画,吾且望之却走矣。”
\end{comments}
\end{yulu}

\begin{yulu}{读书与阅历}
少年读书,如隙中窥月;中年读书,如庭中望月;老年读书,如台上玩月。皆以阅历之浅深,为所得之浅深耳。
\begin{comments}
黄交三曰:“真能知读书痛痒者也。” \\
张竹坡曰:“吾叔此论,直置身广寒宫里,下视大千世界,皆清光似水矣。” \\
毕右万曰:“吾以为学道亦有浅深之别。”
\end{comments}
\end{yulu}

\begin{yulu}{致雨师书}
吾欲致书雨师:春雨,宜始于上元节后〈观灯已毕〉,至清明十日前之内〈雨止桃开〉,及谷雨节中;夏雨,宜于每月上弦之前,及下弦之后〈免碍于月〉;秋雨,宜于孟秋、季秋之上下二旬〈八月为玩月胜境〉;至若三冬,正可不必雨也。
\begin{comments}
孔东塘曰:“君若果有此牍,吾愿作致书邮也。” \\
余生生曰:“使天而雨粟,虽自元旦雨至除夕,亦未为不可。” \\
张竹坡曰:“此书独不可致于巫山雨师。”
\end{comments}
\end{yulu}

\begin{yulu}{贫富忧乐}
为浊富不若为清贫,以忧生不若以乐死。
\begin{comments}
李圣许曰:“顺理而生,虽忧不忧;逆理而死,虽乐不乐。” \\
吴野人曰:“我宁愿为浊富。” \\
张竹坡曰:“我愿太奢,欲为清富,焉能遂愿!”
\end{comments}
\end{yulu}

\begin{yulu}{天下唯鬼最富最尊}
天下唯鬼最富,生前囊无一文,死后每饶楮镪;天下唯鬼最尊,生前或受欺凌,死后必多跪拜。
\begin{comments}
吴野人曰:“世于贫士,辄目为‘穷鬼’,则又何也?” \\
陈康畴曰:“穷鬼若死,即幷称尊矣。”
\end{comments}
\end{yulu}

\begin{yulu}{蝶与花}
蝶为才子之化身,花乃美人之别号。
\begin{comments}
张竹坡曰:“蝶入花房香满衣,是反以金屋贮才子矣。”
\end{comments}
\end{yulu}

\begin{yulu}{五“想”者}
因雪想高士;因花想美人;因酒想侠客;因月想好友;因山水想得意诗文。
\begin{comments}
弟木山曰:“余每见人一长、一技,即思效之;虽至琐屑,亦不厌也。大约是爱博而情不专。” \\
张竹坡曰:“多情语令人泣下。” \\
尤谨庸曰:“因得意诗文,想心斋矣。” \\
李季子曰:“此善于设想者。” \\
陆云士曰:“临川谓:‘想内成,因中见。’与此相发。”
\end{comments}
\end{yulu}

\begin{yulu}{闻声如临其境}
闻鹅声如在白门;闻橹声如在三吴;闻滩声如在浙江;闻羸马项下铃铎声,如在长安道上。
\begin{comments}
聂晋人曰:“南无观世音菩萨摩诃萨。” \\
倪永清曰:“众音寂灭时,又作么生话会?”
\end{comments}
\end{yulu}

\begin{yulu}{一岁诸节}
一岁诸节,以上元为第一,中秋次之,五日、九日又次之。
\begin{comments}
张竹坡曰:“一岁当以我畅意日为佳节。” \\
顾天石曰:“跻上元于中秋之上,未免尚耽绮习。”
\end{comments}
\end{yulu}

\begin{yulu}{雨令昼短夜长}
雨之为物,能令昼短;能令夜长。
\begin{comments}
张竹坡曰:“雨之为物,能令天闭眼,能令地生毛,能为水国广封疆。”
\end{comments}
\end{yulu}

\begin{yulu}{古之失传者}
古之不传于今者,啸也,剑术也,弹棋也,打球也。
\begin{comments}
黄九烟曰:“古之绝胜于今者,官妓、女道士也。” \\
张竹坡曰:“今之绝胜于古者,能吏也,猾棍也,无耻也。” \\
庞天池曰:“今之必不能传于后者,八股也。”
\end{comments}
\end{yulu}

\begin{yulu}{诗僧多,诗道少}
诗僧时复有之,若道士之能诗者,不啻空谷足音,何也?
\begin{comments}
毕右万曰:“僧、道能诗,亦非难事;但惜僧、道不知禅玄耳。” \\
顾天石曰:“道于三教中,原属第三。应是根器最钝人做,那得会诗!轩辕弥明,昌黎寓言耳。” \\
尤谨庸曰:“僧家势利第一,能诗次之。” \\
倪永清曰:“我所恨者,辟穀之法不传。”
\end{comments}
\end{yulu}

\begin{yulu}{甯为萱草毋为杜鹃}
当为花中之萱草;毋为鸟中之杜鹃。
\begin{comments}
袁翔甫补评曰:“萱草忘忧,杜鹃啼血。悲欢哀乐,何去何从?”
\end{comments}
\end{yulu}

\begin{yulu}{物穉可厌者}
物之穉者,皆不可厌,惟驴独否。
\begin{comments}
黄略似曰:“物之老者,皆可厌;惟松与梅则否。” \\
倪永清曰:“惟癖于驴者,则不厌之。”
\end{comments}
\end{yulu}

\begin{yulu}{耳闻不如目见}
女子自十四五岁至二十四五岁,此十年中,无论燕、秦、吴、越,其音大都娇媚动人。一睹其貌,则美恶判然矣。耳闻不如目见,于此益信。
\begin{comments}
吴听翁曰:“我曏以耳根之有馀,补目力之不足;今读此,乃知卿言亦复佳也。” \\
江含徵曰:“帘为妓衣,亦殊有见。” \\
张竹坡曰:“家有少年丑婢者,当令隔屏私语,灭烛侍寝。何如?” \\
倪永清曰:“若逢关貌而恶声者,又当何如?”
\end{comments}
\end{yulu}

\begin{yulu}{仙寻乐,佛避苦}
寻乐境乃学仙,避苦趣乃学佛。佛家所谓“极乐世界”者,盖谓众苦之所不到也。
\begin{comments}
江含徵曰:“著败絮行荆棘中,固是苦事;彼披忍辱铠者,亦未得优游自到也。” \\
陆云士曰:“空诸所有,受即是空。其为苦乐,不足言矣,故学佛优于学仙。”
\end{comments}
\end{yulu}

\begin{yulu}{闲贫与谦富}
富贵而劳悴,不若安闲之贫贱;贫贱而骄傲,不若谦恭之富贵。
\begin{comments}
曹实庵曰:“富贵而又安闲,自能谦恭也。” \\
许师六曰:“富贵而又谦恭,乃能安闲耳。” \\
张竹坡曰:“谦恭安闲,乃能长富贵也。” \\
张迂庵曰:“安闲乃能骄傲,劳悴则必谦恭。”
\end{comments}
\end{yulu}

\begin{yulu}{耳能自闻}
目不能自见,鼻不能自嗅,舌不能自舐,手不能自握,惟耳能自闻其声。
\begin{comments}
弟木山曰:“岂不闻心不在焉、听而不闻乎?兄其诳我哉!” \\
张竹坡曰:“心能自信。” \\
释师昂曰:“古德云:眉与目不相识,衹为太近。”
\end{comments}
\end{yulu}

\begin{yulu}{远近皆宜之声}
凡声皆宜远听,惟听琴则远近皆宜。
\begin{comments}
王名友曰:“松涛声、瀑布声、箫笛声、潮声、读书声、钟声、梵声,皆宜远听;惟琴声、度曲声、雪声,非至近不能得其离合抑扬之妙。” \\
庞天池曰:“凡色皆宜近看,惟山色远近皆宜。”
\end{comments}
\end{yulu}

\begin{yulu}{苦闷甚盲哑者}
目不能识字,其闷尤过于盲;手不能执管,其苦更甚于哑。
\begin{comments}
陈鹤山曰:“君独未知今之不识字、不握管者,其乐尤过于不盲、不哑者也。”
\end{comments}
\end{yulu}

\begin{yulu}{人间极乐事}
并头联句,交颈论文,宫中应制,历使属国,皆极人间乐事。
\begin{comments}
狄立人曰:“既已幷头、交颈,即欲联句、论文,恐亦有所不暇。” \\
汪舟次曰:“历使属国,殊不易易。” \\
孙松坪曰:“邯郸旧梦,对此惘然。” \\
张竹坡曰:“幷头、交颈,乐事也;联句、论文,亦乐事也。是以两乐幷为一乐者,则当以两夜幷一夜方妙。然其乐一刻,胜于一日矣。” \\
沈契掌曰:“恐天亦见妒。”
\end{comments}
\end{yulu}

\begin{yulu}{姓之佳劣}
《水浒传》武松诘蒋门神云:“为何不姓李。”此语殊妙。盖姓实有佳有劣:如华、如柳、如雲、如苏、如乔,皆极风韵;若夫毛也、赖也、焦也、牛也,则皆尘于目而棘于耳也。
\begin{comments}
先渭求曰:“然则君为何不姓李耶?” \\
张竹坡曰:“止闻今张昔李,不闻今李昔张也。”
\end{comments}
\end{yulu}

\begin{yulu}{花之宜目宜鼻者}
花之宜于目而复宜于鼻香,梅也、菊也、兰也、水仙也、珠兰也、莲也。止宜于鼻者,橼也、桂也、瑞香也、梔子也、茉莉也、木香也、玫瑰也、腊梅也。馀则皆宜于目者也。花与叶俱可观者,秋海棠为最,荷次之。海棠、酴醾、虞美人、水仙,又次之。叶胜于花者,止雁来红、美人蕉而已。花与叶俱不足观者,紫薇也、辛夷也。
\begin{comments}
周星远曰:“山老可当花阵一面。” \\
张竹坡曰:“以一叶而能胜诸花者,此君也。”
\end{comments}
\end{yulu}

\begin{yulu}{高语山林者当不读史}
高语山林者,辄不喜谈市朝事。审若此,则当并废《史》、《汉》诸书而不读矣。盖诸书所载者,皆古之市朝也。
\begin{comments}
张竹坡曰:“高语者,必是虚声处士;真入山者,方能经纶市朝。”
\end{comments}
\end{yulu}

\begin{yulu}{惟雲不能画}
雲之为物,或崔嵬如山,或潋滟如水;或如人,或如兽;或如鸟毳,或如鱼鳞。故天下万物皆可画,惟雲不能画;世所画雲,亦强名耳。
\begin{comments}
何蔚宗曰:“天下百官皆可做,惟教官不可做。做教官者,皆谪戍耳。” \\
张竹坡曰:“云有反面、正面,有阴阳、向背,有层次、内外。细观其与日相映,则知其明处乃一面,暗处又一面。尝谓古今无一画云手,不谓《幽梦影》中先得我心。”
\end{comments}
\end{yulu}

\begin{yulu}{人生全福}
值太平世,生湖山郡,官长廉静,家道优裕,娶妇贤淑,生子聪慧。人生如此,可云全福。
\begin{comments}
许筱林曰:“若以粗笨愚蠢之人当之,则负却造物。” \\
江含徵曰:“此是黑面老子要思量做鬼处。” \\
吴岱观曰:“过屠门而大嚼,虽不得肉,亦且快意。” \\
李荔园日:“贤淑、聪慧,尤贵永年,否则福不全。”
\end{comments}
\end{yulu}

\begin{yulu}{制工而民贫}
天下器玩之类,其制日工,其价日贱,毋惑乎民之贫也。
\begin{comments}
张竹坡曰:“由于民贫,故益工;而益贱,若不贫,如何肯贱?”
\end{comments}
\end{yulu}

\begin{yulu}{花瓶宜相称}
养花胆瓶,其式之高低大小,须与花相称。而色之浅深浓淡,又须与花相反。
\begin{comments}
程穆倩曰:“足补袁中郎《甁史》所未逮。” \\
张竹坡曰:“夫如此,有不甘去南枝而生香于几案之右者乎!名花心足矣。” \\
王宓草曰:“须知相反者,正欲其相称也。”
\end{comments}
\end{yulu}

\begin{yulu}{春雨、夏雨、秋雨}
春雨如恩诏,夏雨如赦书,秋雨如挽歌。
\begin{comments}
张谐石曰:“我辈居恒苦饥,但愿夏雨如馒头耳。” \\
张竹坡曰:“赦书太多,亦不甚妙。”
\end{comments}
\end{yulu}

\begin{yulu}{全人}
十岁为神童,二十、三十为才子,四十、五十为名臣,六十为神仙,可谓全人矣。
\begin{comments}
江含征曰:“此却不可知,盖神童原有仙骨故也,衹恐中间做名臣时,堕落名利场中耳。” \\
杨圣藻曰:“人孰不想?难得有此全福!” \\
张竹坡曰:“神童、才子由于己,可能也;名臣由于君,仙由于天,不可必也。” \\
顾天石曰:“六十神仙,似乎太早。”
\end{comments}
\end{yulu}

\begin{yulu}{武中之文与文中之武}
武人不苟战,是为武中之文;文人不迂腐,是为文中之武。
\begin{comments}
梅定九曰:“近日文人不迂腐者颇多,心斋亦其一也。” \\
顾天石曰:“然则心斋直谓之武夫可乎?笑笑。” \\
王司直曰:“是真文人,必不迂腐。”
\end{comments}
\end{yulu}

\begin{yulu}{文人讲武与武将论文}
文人讲武事,大都纸上谈兵;武将论文章,半属道听途说。
\begin{comments}
吴街南曰:“今之武将讲武事,亦属纸上谈兵;今之文人论文章,大都道听途说。”
\end{comments}
\end{yulu}

\begin{yulu}{三种斗方可取}
斗方止三种可取:佳诗文一也,新题目二也,精款式三也。
\begin{comments}
闵宾连日:“近年斗方名士甚多,不知能入吾心斋彀中否也?”
\end{comments}
\end{yulu}

\begin{yulu}{情与才}
情必近于痴而始真,才必兼乎趣而始化。
\begin{comments}
陆云士曰:“真情种真才子能为此言。” \\
顾天石曰:“才兼乎趣,非心斋不足当之。” \\
尤慧珠曰:“余情而痴则有之,才而趣则未能也。”
\end{comments}
\end{yulu}

\begin{yulu}{花瓣双全者}
凡花色之娇媚者,多不甚香;瓣之千层者,多不结实。甚矣全才之难也。兼之者,其惟莲乎?
\begin{comments}
殷日戒曰:“花、叶、根、实无所不空,亦无不适于用,莲则全有其德者也。” \\
贯玉曰:“莲花易谢,所谓有全才而无全福也。” \\
王丹麓曰:“我欲荔枝有好花,牡丹有佳实,方妙。” \\
尤谨庸曰:“全才必为人所忌,莲花故名君子。”
\end{comments}
\end{yulu}

\begin{yulu}{著书与注书}
著得一部新书,便是千秋大业;注得一部古书,允为万世宏功。
\begin{comments}
黄交三曰:“世间难事,注书第一。大要于极寻常书,要看出作者苦心。” \\
张竹坡曰:“注书无难,天使人得安居无累,有可以注书之时与地为难耳。”
\end{comments}
\end{yulu}

\begin{yulu}{三“无是处”者}
延名师,训子弟;入名山,习举业;丐名士,代捉刀。三者都无是处。
\begin{comments}
陈康畴曰:“大抵名而已矣,好歹原未必著意。” \\
殷日戒曰:“况今之所谓名乎!”
\end{comments}
\end{yulu}

\begin{yulu}{明来未见新创文体}
积画以成字,积字以成句,积句以成篇,谓之文。文体日增,至八股而遂止。如古文、如诗、如赋、如词、如曲、如说部、如传奇小说,皆自无而有。方其未有之时,固不料后来之有此一体也。逮既有此一体之后,又若天造地设,为世必应有之物。然自明以来,未见有创一体裁新人耳目者。遥计百年之后,必有其人,惜乎不及见耳。
\begin{comments}
陈康畴曰:“天下事,从意起。山来今日既作此想,安知其来生不即为此辈翻新之士乎!惜乎今人不及知耳。” \\
陈鹤山曰:“此是先生应以创体身得度者,即现创体身而为设法。” \\
孙恺似曰:“读心斋别集,拈四子书题,以五七言韵体行之,无不入妙,叹其独绝。此则直可当先生自序也。” \\
张竹坡曰:“见及于此,是必能创之者。吾拭目以待新裁。”
\end{comments}
\end{yulu}

\begin{yulu}{友道可贵}
云映日而成霞,泉挂岩而成瀑。所托者异,而名亦因之。此友道之所以可贵也。
\begin{comments}
张竹坡曰:“非日而云不映,非岩而泉不挂。此友道之所以当择也。”
\end{comments}
\end{yulu}

\begin{yulu}{学大家文与学名家文}
大家之文,吾爱之、慕之,吾愿学之;名家之文,吾爱之、慕之,吾不敢学之。学大家而不得,所谓刻鹄不成,尚类鹜也;学名家而不得,则是画虎不成,反类狗矣。
\begin{comments}
黄旧樵曰:“我则异于是,最恶世之貌为大家者。” \\
殷日戒曰:“彼不曾闯其藩篱,乌能窥其阃奥!衹说得隔壁话耳。” \\
张竹坡曰:“今人读得一两句名家,便自称大家矣。”
\end{comments}
\end{yulu}

\begin{yulu}{二氏之旨}
由戒得定,由定得慧,勉强渐近自然;炼精化气,炼气化神,清虚有何渣滓。
\begin{comments}
袁中江曰:“此二氏之学也。吾儒何独不然!” \\
陆云士曰:“《楞严经》、《参同契》精义尽涵在内。” \\
尤悔庵曰:“极平常语,然道在是矣。”
\end{comments}
\end{yulu}

\begin{yulu}{一定之位与无定之位}
南北东西,一定之位也;前后左右,无定之位也。
\begin{comments}
张竹坡曰:“闻天地昼夜旋转,则此东西南北,亦无定之位也。或者天地外贮此天地者,当有一定耳。”
\end{comments}
\end{yulu}

\begin{yulu}{二氏不可废}
予尝谓二氏不可废,非袭夫大养济院之陈言也。盖名山胜景,我辈每思蹇裳就之。使非琳宫梵刹,则倦时无可驻足,饥时谁与授餐。忽有疾风暴雨,五大夫果真足恃乎?又或丘壑深邃,非一日可了,岂能露宿以待明日乎?虎豹蛇虺,能保其不人患乎?又或为士大夫所有,果能不问主人,任我之登陟凭吊而莫之禁乎?不特此也。甲之所有,乙思起而夺之,是启争端也。祖父之所创建,子孙贫力不能修葺,其倾颓之状,反足令山川减色矣!然此特就名山胜境言之耳。即城市之内,与夫四达之衢,亦不可少此一种。客游可作居停,一也;长途可以稍憩,二也;夏之茗、冬之薑汤,复可以济役夫负戴之困,三也。凡此皆就事理言之,非二氏福报之说也。
\begin{comments}
释中洲曰:“此论一出,量无悭檀越矣。” \\
张竹坡曰:“如此处置此辈甚妥。但不得令其于人家丧事诵经,吉事拜忏;装金为像,铸铜作身;房如宫殿,器御钟鼓,动说因果;虽饮酒食肉,娶妻生子,总无不可。” \\
石天外曰:“天地生气,大抵五十年一聚。生气一聚,必有刀兵、饥馑、瘟疫以收其生气。此古今一治一乱,必然之数也。自佛入中国,用剃度出家法,绝其后嗣,天地盖欲以佛节古今之生气也。所以,唐、宋、元、明以来,剃度者多,而刀兵去刀数稍减于春秋、战国、秦、汉诸时也。然则佛氏且未必无功于天地,宁特人类已哉!”
\end{comments}
\end{yulu}

\begin{yulu}{三“不可不”者}
虽不善书,而笔砚不可不精;虽不业医,而验方不可不存;虽不工弈,而楸枰不可不备。
\begin{comments}
江含徵曰:“虽不善饮,而良醖不可不藏,此坡仙之所以为坡仙也。” \\
顾天石曰:“虽不好色,而美女妖童不可不蓄。” \\
毕右万曰:“虽不习武,而弓矢不可不张。”
\end{comments}
\end{yulu}

\begin{yulu}{方外须戒俗,红裙须得趣}
方外不必戒酒,但须戒俗;红裙不必通文,但须得趣。
\begin{comments}
朱其恭曰:“以不戒酒之方外,遇不通文之红裙,必有可观。” \\
陈定九曰:“我不善饮,而方外不饮酒者誓不与之语;红裙若不识趣,亦不乐与近。” \\
释浮村曰:“得居士此论,我辈可放心豪饮矣。” \\
弟东囿曰:“方外幷戒了化缘,方妙。”
\end{comments}
\end{yulu}

\begin{yulu}{石之“宜”者}
梅边之石宜古;松下之石宜拙;竹傍之石宜瘦;盆内之石宜巧。
\begin{comments}
周星远曰:“论石至此,直可作九品中正。” \\
释中洲曰:“位置相当,足见胸次。”
\end{comments}
\end{yulu}

\begin{yulu}{律己与处世}
律己宜带秋气;处世宜带春气。
\begin{comments}
孙松楸曰:“君子所以有矜群而无争党也。” \\
胡静夫曰:“合夷、惠为一人,吾愿亲炙之。” \\
尤悔庵曰:“皮里春秋。”
\end{comments}
\end{yulu}

\begin{yulu}{催租与布施}
厌催租之败意,亟宜早早完粮;喜老衲之谈禅,难免常常布施。
\begin{comments}
释中洲曰:“居士辈之实情,吾僧家之私冀,直被一笔写出矣。” \\
瞎尊者曰:“我不会谈禅,亦不敢妄求布施,惟闲写青山卖耳。”
\end{comments}
\end{yulu}

\begin{yulu}{耳听诸音}
松下听琴;月下听箫;涧边听瀑布;山中听梵呗,觉耳中别有不同。
\begin{comments}
张竹坡曰:“其不同处,有难于向不知者道。” \\
倪永清曰:“识得‘不同’二字,方许享此清听。”
\end{comments}
\end{yulu}

\begin{yulu}{月下韵事}
月下谈禅,旨趣益远;月下说剑,肝胆益真;月下论诗,风致益幽;月下对美人,情意益笃。
\begin{comments}
袁士旦曰:“溽暑中赴华筵,冰雪中应考试,阴雨中对道学先生,与此况味何如?”
\end{comments}
\end{yulu}

\begin{yulu}{山水之妙}
有地上之山水,有画上之山水,有梦中之山水,有胸中之山水。地上者妙在丘壑深邃;画上者妙在笔墨淋漓;梦中者妙在景象变幻;胸中者妙在位置自如。
\begin{comments}
周星远曰:“心斋《幽梦影》中文字,其妙亦在景象变幻。” \\
殷日戒曰:“若诗文中之山水,其幽深变幻更不可以名状。” \\
江含徵曰:“但不可有面上之山水。” \\
余香祖曰:“余境况不佳,水穷山尽矣。”
\end{comments}
\end{yulu}

\begin{yulu}{一日、一岁、十年、百年之计}
一日之计种蕉;一岁之计种竹;十年之计种柳;百年之计种松。
\begin{comments}
周星远曰:“千秋之计,其著书乎?” \\
张竹坡曰:“百世之计种德。”
\end{comments}
\end{yulu}

\begin{yulu}{四时雨宜事}
春雨宜读书;夏雨宜弈棋;秋雨宜检藏;冬雨宜饮酒。
\begin{comments}
周星远曰:“四时惟秋雨最难听,然予谓无分今雨、旧雨,听之要皆宜于饮也。”
\end{comments}
\end{yulu}

\begin{yulu}{诗文词曲之体}
诗文之体,得秋气为佳;词曲之体,得春气为佳。
\begin{comments}
江含徵曰:“调有惨淡悲伤者,亦须相称。” \\
殷日戒曰:“陶诗,欧文,亦似以春气胜。”
\end{comments}
\end{yulu}

\begin{yulu}{“不必过求”与“不可不求”}
抄写之笔墨,不必过求其佳,若施之缣素,则不可不求其佳;诵读之书籍,不必过求其备,若以供稽考,则不可不求其备;游历之山水,不必过求其妙,若因之卜居,则不可不求其妙。
\begin{comments}
冒辟疆曰:“外遇之女色,不必过求其美;若以作姬妾,则不可不求其美。” \\
倪永清曰:“观其区处条理,所在经济可知。” \\
王司直曰:“求其所当求,而不求其所不必求。”
\end{comments}
\end{yulu}

\begin{yulu}{求知}
人非圣贤,安能无所不知。祇知其一,惟恐不止其一,复求知其二者,上也;止知其一,因人言,始知有其二者,次也;止知其一,人言有其二而莫之信者,又其次也;止知其一,恶人言有其二者,斯下之下矣。
\begin{comments}
周星远曰:“兼听则聪,心斋所以深于知也。” \\
倪永清曰:“圣贤大学问,不意于清语得之。”
\end{comments}
\end{yulu}

\begin{yulu}{横直世界}
史官所纪者,直世界也;职方所载者,横世界也。
\begin{comments}
袁中江曰:“众宰官所治者,斜世界也。” \\
尤悔庵曰:“普天下所行者,混沌世界也。” \\
顾天石曰:“吾尝思天上之天堂,何处筑基?地下之地狱,何处出气?世界固有不可思议者!”
\end{comments}
\end{yulu}

\begin{yulu}{先天、后天八卦}
先天八卦,竖看者也;后天八卦,横看者也。
\begin{comments}
吴街南曰:“横看竖看,皆看不著。” \\
钱目天曰:“何如袖手旁观!”
\end{comments}
\end{yulu}

\begin{yulu}{书之藏、看、读、用、记}
藏书不难,能看为难;看书不难,能读为难;读书不难,能用为难;能用不难,能记为难。
\begin{comments}
洪去芜曰:“心斋以‘能记’次于‘能用’之后,想亦苦记性不如耳。世固有能记而不能用者。” \\
王端人曰:“能记、能用,方是真藏书人。” \\
张竹坡曰:“能记固难,能行尤难。”
\end{comments}
\end{yulu}

\begin{yulu}{求知己难}
求知己于朋友易,求知己于妻妾难,求知己于君臣则尤难之难。
\begin{comments}
王名友曰:“求知己于妾易,求知己于妻难,求知己于有妾之妻尤难。” \\
张竹坡曰:“求知己于兄弟亦难。” \\
江含徵曰:“求知己于鬼神则反易耳。”
\end{comments}
\end{yulu}

\begin{yulu}{善人与恶人}
何谓善人,无损于世者则谓之善人;何谓恶人,有害于世者则谓之恶人。
\begin{comments}
江含徵曰:“尚有有害于世而反邀善人之誉,此实为好利而显为名高者,则又恶人之尤。”
\end{comments}
\end{yulu}

\begin{yulu}{五福}
有工夫读书,谓之福;有力量济人,谓之福;有学问著述,谓之福;无是非到耳,谓之福;有多闻直谅之友,谓之福。
\begin{comments}
殷日戒曰:“我本薄福人,宜行求福事,在随时儆醒而已。” \\
杨圣藻曰:“在我者可必,在人者不能必。” \\
王丹麓曰:“备此福者,惟我心斋。” \\
李水樵曰:“五福骈臻固佳,苟得其半者,亦不得谓之无福。” \\
倪永清曰:“直谅之友,富贵人久拒之矣,何心斋反求之也?”
\end{comments}
\end{yulu}

\begin{yulu}{至乐莫大于闲}
人莫乐于闲,非无所事事之谓也。闲则能读书,闲则能游名胜,闲则能交益友,闲则能饮酒,闲则能著书。天下之乐,孰大于是。
\begin{comments}
陈鹤山曰:“然则正是极忙处。” \\
黄交三曰:“‘闲’字前,有止敬功夫,方能到此。” \\
尤悔庵曰:“昔人云‘忙里偷闲’,闲而可偷,盗亦有道矣。” \\
李若金曰:“闲固难得。有此五者,方不负‘闲’字。”
\end{comments}
\end{yulu}

\begin{yulu}{文章与山水}
文章是案头之山水,山水是地上之文章。
\begin{comments}
李圣许曰:“文章必明秀,方可作案头山水;山水必曲折,乃可名地上文章。”
\end{comments}
\end{yulu}

\begin{yulu}{凡字非皆有四声}
平上去入,乃一定之至理。然入声之为字也少,不得谓凡字皆有四声也。世之调平仄者,于入声之无其字者,往往以不相合之音隶于其下。为所隶者,苟无平上去之三声,则是以寡妇配鳏夫,犹之可也;若所隶之字自有其平上去之三声,而欲强以从我,则是干有夫之妇矣,其可乎?姑就诗韵言之,如“东”、“冬”韵,无入声者也,今人尽调之以“东”、“董”、“冻”、“督”。夫“督”之为音,当附于“都”、“睹”、“妒”之下;若属之于“东”、“董”、“冻”,又何以处夫“都”、“睹”、“妒”乎?若“东”、“都”二字俱以“督”字为入声,则是一妇而两夫矣。三“江”无入声者也,今人尽调之以“江”、“讲”、“绛”、“觉”,殊不知“觉”之为音,当附于“交”、“教”之下者也。诸如此类,不胜其举。然则如之何而后可?曰:鳏者听其鳏,寡者听其寡,夫妇全者安其全,各不相干而已矣。〈“东”、“冬”、“欢”、“桓”、“寒”、“山”、“真”、“文”、“元”、“渊”、“先”、“天”、“庚”、“青”、“侵”、“盐”、“咸”诸部,皆无入声者也。“屋”、“沃”内如“秃”、“独”、“鹄”、“束”等字,乃“鱼”、“虞”韵内“都”、“图”等字之入声;“卜”、“木”、“六”、“仆”等字,乃五“歌”部之入声;“玉”、“菊”、“狱”、“育”等字,乃“尤”部之入声;三“觉”、十“药”,当属于“萧”、“肴”、“豪”;“质”、“锡”、“职”、“缉”,当属于“支”、“微”、“齐”。“质”内之“橘”、“卒”,“物”内之“郁”、“屈”,当属于“虞”、“鱼”;“物”内之“勿”、“物”等音,无平上去者也;“讫”、“乞”等,四“支”之入声也。“陌”部乃“佳”、“灰”之半,“开”、“来”等字之入声也。“月”部之“月”、“厥”、“谒”等及“屑”、“葉”二部,古无平上去,而今则为中州韵内“车”、“遮”诸字之入声也。“伐”、“髮”等字及“曷”部之“括”、“适”及八“黠”全部,又十五“合”内诸字,又十七“洽”全部,皆六“麻”之入声也。“曷”内之“撮”、“阔”等字,“合”部之“合”、“盒”数字,皆无平上去者也。若以“缉”、“合”、“葉”、“洽”为闭口韵,则止当谓之无平上去之寡妇,而不当调之以“侵”、“寝”。“缉”、“咸”、“喊”、“陷”、“洽”也。〉
\begin{comments}
石天外曰:“中州韵无入声,是有夫无妇,天下皆成旷夫世界矣。”
\end{comments}
\end{yulu}

\begin{yulu}{怒悟哀三书}
《水浒传》是一部怒书;《西游记》是一部悟书;《金瓶梅》是一部哀书。
\begin{comments}
江含徵曰:“不会看《金甁梅》,而衹学其淫,是爱东坡者,但喜喫东坡肉耳。” \\
殷日戒曰:“《幽梦影》是一部快书。” \\
朱其恭曰:“余谓《幽梦影》是一部趣书。”
\end{comments}
\end{yulu}

\begin{yulu}{读史多怒亦乐处}
读书最乐,若读史书,则喜少怒多,究之怒处亦乐处也。
\begin{comments}
张竹坡曰:“读到喜、怒俱忘,是大乐境。” \\
陆云士曰:“余尝有句云:‘读《三国志》,无人不为刘;读《南宋书》,无人不冤岳。’第人不知怒处亦乐处耳。怒而能乐,惟善读史者知之。”
\end{comments}
\end{yulu}

\begin{yulu}{奇书与密友}
发前人未发之论,方是奇书;言妻子难言之情,乃为密友。
\begin{comments}
孙恺似曰:“前二语,是心斋著书本领。” \\
毕右万曰:“奇书我却有数种,如人不肯看何?” \\
陆云士曰:“《幽梦影》一书所发者,皆未发之论;所言者,皆难言之情。欲语羞雷同,可以题赠。”
\end{comments}
\end{yulu}

\begin{yulu}{一介之士,必有密友}
一介之士,必有密友。密友不必定是刎颈之交,大率虽千百里之遥,皆可相信,而不为浮言所动;闻有谤之者,即多方为之辩析而后已;事之宜行宜止者,代为筹画决断;或事当利害关头,有所需而后济者,即不必与闻,亦不虑其负我与否,竟为力承其事。此皆所谓密友也。
\begin{comments}
殷日戒曰:“后段更见恳切周详,可以想见其为人矣。” \\
石天外曰:“如此密友,人生能得几个?仆愿心斋先生当之。”
\end{comments}
\end{yulu}


\chapter{卷下}

\begin{yulu}{风流真率}
风流自赏,祇容花鸟趋陪;真率谁知,合受烟霞供养。
\begin{comments}
江含徵曰:“东坡有云:‘当此之时,若有所思,而无所思。’”
\end{comments}
\end{yulu}

\begin{yulu}{名心难忘,美酒未淡}
万事可忘,难忘者名心一段;千般易淡,未淡者美酒三杯。
\begin{comments}
张竹坡曰:“是闻鸡起舞,酒后耳热气象。” \\
王丹麓曰:“予性不耐饮,美酒亦易淡。所最难忘者,名耳。” \\
陆云士曰:“惟恐不好名。丹麓此言,具见真处。”
\end{comments}
\end{yulu}

\begin{yulu}{芰荷金石}
芰荷可食而亦可衣,金石可器而亦可服。
\begin{comments}
张竹坡曰:“然后知濂溪不过为衣食计耳。” \\
王司直曰:“今之为衣食计者,果似濂溪否?”
\end{comments}
\end{yulu}

\begin{yulu}{宜耳宜目者}
宜于耳复宜于目者,弹琴也,吹箫也。宜于耳不宜于目者,吹笙也,擫管也。
\begin{comments}
李圣许曰:“宜于目不宜于耳者,狮子吼之美妇人也;不宜于目幷不宜于耳者,面目可憎、语言无味之纨袴子也。” \\
庞天池曰:“宜于耳复宜于目者,巧言令色也。”
\end{comments}
\end{yulu}

\begin{yulu}{宜看晓妆时}
看晓妆宜于傅粉之后。
\begin{comments}
余淡心曰:“看晚妆,不知心斋以为宜于何时?” \\
周冰持曰:“不可说!不可说!” \\
黄交三曰:“水晶帘下看梳头,不知尔时曾傅粉否?” \\
庞天池曰:“看残妆,宜于微醉后,然眼花撩乱矣。”
\end{comments}
\end{yulu}

\begin{yulu}{千古相思者}
我不知我之生前,当春秋之季,曾一识西施否?当典午之时,曾一看卫玠否?当义熙之世,曾一醉渊明否?当天宝之代,曾一睹太真否?当元丰之朝,曾一晤东坡否?千古之上,相思者不止此数人,而此数人则其尤甚者,故姑举之以概其馀也。
\begin{comments}
杨圣藻曰:“君前生曾与诸君周旋,亦未可知,但今生忘之耳。” \\
纪伯紫曰:“君之前生,或竞是渊明、东坡诸人,亦未可知。” \\
王名友曰:“不特此也!心斋自云:‘愿来生为绝代佳人!’又安知西施、太真不即为其前生耶!” \\
郑破水曰:“赞叹爱慕,千古一情。美人不必为妻妾,名士不必为朋友,又何必问之前生也耶!心斋真情痴也。” \\
陆云士曰:“余尝有诗曰:‘自昔闻佛言,人有轮回事。前生为古人,不知何姓氏!或览青史中,若与他人遇!’竟与心斋同情,然大逊其奇快!”
\end{comments}
\end{yulu}

\begin{yulu}{交谈高士名妓}
我又不知在隆万时,曾于旧院中交几名妓,眉公伯虎若士赤水诸君,曾共我谈笑几回。茫茫宇宙,我今当向谁问之耶!
\begin{comments}
江含徵曰:“死者有知,则良晤匪遥。如各化为异物,吾未如之何也已!” \\
顾天石曰:“具此襟情,百年后当有恨不与心斋周旋者,则吾幸矣!”
\end{comments}
\end{yulu}

\begin{yulu}{文章与锦绣}
文章是有字句之锦绣,锦绣是无字句之文章。两者同出于一原,姑即粗迹论之,如金陵、如武林、如姑苏,书林之所在,即杼机之所在也。
\begin{comments}
袁翔甫补评曰:“若兰回文,是有字句之锦绣也;落花水面,是无字句之文章也。”
\end{comments}
\end{yulu}

\begin{yulu}{《千字文》犹未备诗家常用字}
予尝集诸法帖字,为诗字之不复而多者,莫善于《千字文》。然诗家目前常用之字,犹苦其未备。如天文之“烟霞风雪”,地理之“江山塘岸”,时令之“春霄晓暮”,人物之“翁僧渔樵”,花木之“花柳苔萍”,鸟兽之“蜂蝶莺燕”,宫室之“台槛轩窗”,器用之“舟船壶杖”,人事之“梦忆愁恨”,衣服之“裙袖锦绮”,饮食之“茶浆饮酌”,身体之“鬚眉韵态”,声色之“红绿香艳”,文史之“骚赋题吟”,数目之“一三双半”,皆无其字。《千字文》且然,况其他乎!
\begin{comments}
黄仙裳曰:“山来此种诗,竟似为我而设。” \\
顾天石曰:“使其皆备,则《千字文》不为奇矣!吾尝于千字之外另集千字,而已不可复得,更奇。”
\end{comments}
\end{yulu}

\begin{yulu}{三“不可见”者}
花不可见其落,月不可见其沉,美人不可见其夭。
\begin{comments}
朱其恭曰:“君言谬矣!洵如所云,则美人必见其发白齿豁,而后快耶!”
\end{comments}
\end{yulu}

\begin{yulu}{四“须见”者}
种花须见其开,待月须见其满,著书须见其成,美人须见其畅适,方有实际。否则皆为虚设。
\begin{comments}
王璞庵曰:“此条与上条互相发明。盖曰‘花不可见其落’耳,必须见其开也。”
\end{comments}
\end{yulu}

\begin{yulu}{恨不见不传之书}
惠施多方,其书五车。虞卿以穷愁著书,今皆不传。不知书中果作何语?我不见古人,安得不恨!
\begin{comments}
王仔园曰:“想亦与《幽梦影》相类耳!” \\
顾天石曰:“古人所读之书,所著之书,若不被秦人所烧尽,则奇奇怪怪,可供今人刻画者,知复何限!然如《幽梦影》等书出,不必思古人矣。” \\
倪永清曰:“有著书之名,而不见书,省人多少指摘!” \\
庞天池曰:“我独恨古人不见心斋!”
\end{comments}
\end{yulu}

\begin{yulu}{松之诸用}
以松花为粮,以松实为香,以松枝为麈尾,以松阴为步障,以松涛为鼓吹。山居得乔松百馀章,真乃受用不尽。
\begin{comments}
施愚山曰:“君独不记曾有‘松多大蚁’之恨耶!” \\
江含徵曰:“松多大蚁,不妨便为蚁王。” \\
石天外曰:“坐乔松下,如在水晶宫中见万顷波涛总在头上,真仙境也!”
\end{comments}
\end{yulu}

\begin{yulu}{玩月之法}
玩月之法:皎洁则宜仰观,朦胧则宜俯视。
\begin{comments}
孔东塘曰:“深得玩月三昧。”
\end{comments}
\end{yulu}

\begin{yulu}{甘食悦色为性}
孩提之童,一无所知;目不能辨美恶,耳不能判清浊,鼻不能别香臭,至若味之甘苦,则不第知之,且能之弃之。告子以甘食,悦色为性,殆指此类耳!
\begin{comments}

\end{comments}
\end{yulu}


\begin{yulu}{“宜刻”、“宜贪”、“宜痴”者}
凡事不宜刻,若读书则不可不刻;凡事不宜贪,若买书则不可不贪;凡事不宜痴,若行善则不可不痴。
\begin{comments}
余淡心曰:“‘读书不可不刻’,请去一‘读’字,移以赠我,何如?” \\
张竹坡曰:“我为刻书累,请幷去一‘不’字。” \\
杨圣藻曰:“行善不痴,是邀名矣。”
\end{comments}
\end{yulu}

\begin{yulu}{酒色财气}
酒可好不可骂座,色可好不可伤生,财可好不可昧心,气可好不可越理。
\begin{comments}
袁中江曰:“如灌夫使酒,文园病肺,昨夜南塘一出,马上挟章台柳归,亦自无妨,觉愈见英雄本色也。”
\end{comments}
\end{yulu}

\begin{yulu}{文名、俭德、清闲}
文名可以当科第,俭德可以当货财,清闲可以当寿考。
\begin{comments}
聂晋人曰:“若名人而登甲第,富翁而不骄奢,寿翁而又清闲,便是蓬壶三岛中人也。” \\
范汝受曰:“此亦是贫贱文人无所事事,自为慰藉云耳;恐亦无实在受用处也。” \\
曾青藜曰:“‘无事此静坐,一日似两日。若活七十年,便是百四十。’此是清闲当寿考注脚。” \\
石天外曰:“得《老子》‘退一步’法。” \\
顾天石曰:“予生平喜游,每逢佳山水,辄留连不去,亦自谓‘可当园亭之乐’。质之心斋,以为然否?”
\end{comments}
\end{yulu}

\begin{yulu}{尚友古人}
不独诵其诗、读其书,是尚友古人;即观其字画,亦是尚友古人处。
\begin{comments}
张竹坡曰:“能友字画中之古人,则九原皆为之感泣矣!”
\end{comments}
\end{yulu}

\begin{yulu}{斋僧与祝寿}
无益之施舍,莫过于斋僧;无益之诗文,莫过于祝寿。
\begin{comments}
张竹坡曰:“无益之心思,莫过于忧贫;无益之学问,莫过于务名。” \\
殷简堂曰:“若诗文有笔资,亦未尝不可。” \\
庞天池曰:“有益之施舍,莫过于多送我《幽梦影》几册。”
\end{comments}
\end{yulu}

\begin{yulu}{妻贤与境顺}
妾美不如妻贤,钱多不如境顺。
\begin{comments}
张竹坡曰:“此所谓‘竿头欲进步’者。然妻不贤,安用妾美?钱不多,那得境顺?” \\
张迂庵曰:“此盖谓二者不可得兼,舍一而取一者也。”又曰:“世固有钱多而境不顺者。”
\end{comments}
\end{yulu}

\begin{yulu}{修古庙与温旧业}
创新庵不若修古庙,读生书不若温旧业。
\begin{comments}
张竹坡曰:“是真会读书者,是真读过万卷书者,是真一书曾读过数遍者。” \\
顾天石曰:“惟《左传》、《楚词》、马、班、杜、韩之诗文,及《水浒》、《西厢》、《还魂》等书,虽读百遍不厌。此外皆不耐温者矣,奈何!” \\
王安节曰:“今世建生祠,又不若创茅庵。”
\end{comments}
\end{yulu}

\begin{yulu}{字画同原}
字与画同出一原。观六书始于象形,则可知已。
\begin{comments}
江含徵曰:“有不可画之字,不得不用六法也。” \\
张竹坡曰:“千古人未经道破,却一口拈出。”
\end{comments}
\end{yulu}

\begin{yulu}{园亭与住宅}
忙人园亭,宜与住宅相连;闲人园亭,不妨与住宅相远。
\begin{comments}
张竹坡曰:“真闲人,必以园亭为住宅。”
\end{comments}
\end{yulu}

\begin{yulu}{酒茶、诗文、曲词、灯月、笔口、婢奴}
酒可以当茶,茶不可以当酒;诗可以当文,文不可以当诗;曲可以当词,词不可以当曲;月可以当灯,灯不可以当月;笔可以当口,口不可以当笔;婢可以当奴,奴不可以当婢。
\begin{comments}
江含徵曰:“婢当奴,则太亲,吾恐‘忽闻河东狮子吼’耳!” \\
周星远曰:“奴亦有可以当婢处,但未免稍逊耳。”又曰:近时“士大夫往往耽此癖。吾辈驰骛之流,盗此虚名,亦欲效颦相尚;滔滔者天下皆是也,心斋岂未识其故乎?” \\
张竹坡曰:“婢可以当奴者,有奴之所有者也;奴不可以当婢者,有婢之所同有,无婢之所独有者也。” \\
弟木山曰:“兄于饮食之顷,恐月不可以当灯。” \\
余湘客曰:“以奴当婢,小姐权时落后也。” \\
宗子发曰:“惟帝王家不妨以奴当婢,盖以有阉割法也。每见人家奴子出入主母卧房,亦殊可虑。”
\end{comments}
\end{yulu}

\begin{yulu}{酒剑消不平}
胸中小不平,可以酒消之;世间大不平,非剑不能消也。
\begin{comments}
周星远曰:“‘看剑引杯长。’一切不平皆破除矣。” \\
张竹坡曰:“此平世剑术,非隐娘辈所知。” \\
张迂庵曰:“苍苍者未必肯以太阿假人,似不能代作空空儿也。” \\
尤悔庵曰:“龙泉、太阿,汝知我者,岂止苏子美以一斗读《汉书》耶!”
\end{comments}
\end{yulu}

\begin{yulu}{谀骂宁口毋笔}
不得已而谀之者,宁以口,毋以笔;不可耐而骂之者,亦宁以口,毋以笔。
\begin{comments}
孙豹人曰:“但恐未必能自主耳!” \\
张竹坡曰:“上句立品,下句立德。” \\
张迂庵曰:“匪惟立德,亦以免祸。” \\
顾天石曰:“今人笔不谀人,更无用笔之处矣。心斋不知此苦,还是唐宋以上人耳!” \\
陆云士曰:“古《笔铭》曰:‘毫毛茂茂,陷水可脱,陷文不活。’正此谓也。亦有谀以笔而实讥之者,亦有骂以笔而若誉之者。总以不笔为高。”
\end{comments}
\end{yulu}

\begin{yulu}{多情者、红颜者、能诗者}
多情者必好色,而好色者未必尽属多情;红颜者必薄命,而薄命者未必尽属红颜;能诗者必好酒,而好酒者未必尽属能诗。
\begin{comments}
张竹坡曰:“情起于色者,则好色也,非情也;祸起于颜色者,则薄命在红颜,否则亦止曰命而已矣!” \\
洪秋士曰:“世亦有能诗而不好酒者。”
\end{comments}
\end{yulu}

\begin{yulu}{群芳论}
梅令人高,兰令人幽,菊令人野,莲令人淡,春海棠令人艳,牡丹令人豪,蕉与竹令人韵,秋海棠令人媚,松令人逸,桐令人清,柳令人感。
\begin{comments}
张竹坡曰:“美人令众卉皆香,名士令群芳俱舞。” \\
尤谨庸曰:“读之惊才绝艳,堪采入《群芳谱》中。”
\end{comments}
\end{yulu}

\begin{yulu}{物之能感人者}
物之能感人者,在天莫如月,在乐莫如琴,在动物莫如鹃,在植物莫如柳。
\begin{comments}

\end{comments}
\end{yulu}

\begin{yulu}{妻子与奴婢}
妻子颇足累人,羡和靖梅妻鹤子;奴婢亦能供职,喜志和樵婢渔奴。
\begin{comments}
尤悔庵曰:“‘梅妻鹤子,樵婢渔童’,可称绝对。人生眷属,得此足矣!”
\end{comments}
\end{yulu}

\begin{yulu}{渉猎与清高}
渉猎虽曰无用,犹胜于不通古今;清高固然可嘉,莫流于不识时务。
\begin{comments}
黄交三曰:“南阳抱膝时,原非清高者可比。” \\
江含徵曰:“此是心斋经济语。” \\
张竹坡曰:“不合时宜,则可;不达时务,奚其可?” \\
尤悔庵曰:“名言!名言!”
\end{comments}
\end{yulu}

\begin{yulu}{美人之谓}
所谓美人者:以花为貌,以鸟为声,以月为神,以柳为态,以玉为骨,以冰雪为肤,以秋水为姿,以诗词为心。吾无间然矣。
\begin{comments}
冒辟疆曰:“合古今灵秀之气,庶几铸此一人。” \\
江含徵曰:“还要有松蘖之操才好。” \\
黄交三曰:“论美人而曰‘以诗词为心’,真是闻所未闻!”
\end{comments}
\end{yulu}

\begin{yulu}{蚊蝇以人为何物}
蝇集人面,蚊嘬人肤,不知以人为何物!
\begin{comments}
陈康畴曰:“应是头陀转世,意中但求布施也。” \\
释菌人曰:“不堪道破!” \\
张竹坡曰:“此《南华》精髓也。” \\
尤悔庵曰:“正以人之血肉衹堪供蝇蚊咀嚼耳。以我视之,人也;自蝇蚊视之,何异腥膻臭腐乎!” \\
陆云士曰:“集人面者,非蝇而蝇;嘬人肤者,非蚊而蚊。明知其为人也,而集之、嘬之,更不知其以人为何物!”
\end{comments}
\end{yulu}

\begin{yulu}{乐之不知享、不能享、不善享者}
有山林隐逸之乐而不知享者,渔樵也、农圃也、缁黄也;有园亭姬妾之乐,而不能享、不善享者,富商也、大僚也。
\begin{comments}
弟木山曰:“有山珍海错而不能享者,庖人也;有牙签玉轴而不能读者,蠹鱼也,书贾也。”
\end{comments}
\end{yulu}

\begin{yulu}{物各有偶,儗必于伦}
黎举云:“欲令梅聘海棠,枨子〈想是橙。〉臣樱桃,以芥嫁笋,但时不同耳。”予谓“物各有偶,儗必于伦”。今之嫁娶,殊觉未当。如梅之为物,品最清高;棠之为物,姿极妖艳。即使同时,亦不可为夫妇。不若梅聘梨花,海棠嫁杏,橼臣佛手,荔枝臣樱桃,秋海棠嫁雁来红,庶几相称耳。至若以芥嫁笋,笋如有知,必受河东狮子之累矣。
\begin{comments}
弟木山曰:“余尝以芍药为牡丹后,因作贺表一通。兄曾云:‘但恐芍药未必肯耳!’” \\
石天外曰:“花神有知,当以花果数昇谢蹇修矣。” \\
姜学在曰:“雁来红做新郎,真个是老少年也。”
\end{comments}
\end{yulu}

\begin{yulu}{五色与黑白}
五色有太过、有不及,惟黑与白无太过。
\begin{comments}
杜茶村曰:“君独不闻唐有李太白乎?” \\
江含征曰:“又不闻‘玄之又玄’乎?” \\
尤悔庵曰:“知此道者,其惟弈乎!老子曰:‘知其白,守其黑。’”
\end{comments}
\end{yulu}

\begin{yulu}{许氏《说文》之疵}
许氏《说文》,分部有止有其部,而无所属之字者,下必注云:“凡某之属,皆从某。”赘句殊觉可笑,何不省此一句乎!
\begin{comments}
谭公子曰:“此独民县到任告示耳。” \\
王司直曰:“此亦古史之遗。”
\end{comments}
\end{yulu}

\begin{yulu}{人生快意事}
阅《水浒传》,至鲁达打镇关西、武松打虎,因思人生必有一桩极快意事,方不枉在生一场。即不能有其事,亦须著得一种得意之书,庶几无憾耳。〈如李太白有贵妃捧砚事,司马相如有文君当罏事,严子陵有足加帝腹事,王涣、王昌龄有旗亭画壁事,王子安有顺风过江作《滕王阁序》事之类。〉
\begin{comments}
张竹坡曰:“此等事,必须无意中方做得来。” \\
陆云士曰:“心斋所著得意之书颇多,不止一打快活林、一打景阳冈称快意矣。” \\
弟木山曰:“兄若打中山狼,更极快意。”
\end{comments}
\end{yulu}

\begin{yulu}{四时风}
春风如酒,夏风如茗,秋风如烟,冬风如姜芥。
\begin{comments}
许筠庵曰:“所以‘秋风客’气味狠辣。” \\
张竹坡日:“安得东风夜夜来!”
\end{comments}
\end{yulu}

\begin{yulu}{冰裂纹}
冰裂纹极雅,然宜细、不宜肥,若以之作窗栏,殊不耐观也。〈冰裂纹须分大小,先作大冰裂,再于每大块之中,作小冰裂方佳。〉
\begin{comments}
江含徵曰:“此便是哥窑纹也。” \\
靳熊封曰:“‘一片冰心在玉壶’,可以移赠。”
\end{comments}
\end{yulu}

\begin{yulu}{鸟声次佳者殆高士之俦}
鸟声之最佳者,画眉第一,黄鹂、百舌次之。然黄鹂、百舌,世未有笼而畜之者,其殆高士之俦,可闻而不可屈者耶。
\begin{comments}
江含徵曰:“又有‘打起黄莺儿’者,然则亦有时用他不著。” \\
陆云士曰:“‘黄鹂住久浑相识,欲别频啼四五声。’来去有情,正不必笼而畜之也。”
\end{comments}
\end{yulu}

\begin{yulu}{累人与累己}
不治生产,其后必致累人;专务交游,其后必致累己。
\begin{comments}
杨圣藻曰:“晨钟夕磬,发人深省。” \\
冒巢民曰:“若在我,虽累己、累人,亦所不悔。” \\
宗子发曰:“累己犹可,若累人,则不可矣。” \\
江含徵曰:“今之人未必肯受你累,还是自家稳些的好。”
\end{comments}
\end{yulu}

\begin{yulu}{识字无过}
昔人云:“妇人识字,多致诲淫。”予谓此非识字之过也。盖识字则非无闻之人,其淫也,人易得而知耳。
\begin{comments}
张竹坡曰:“此名士持身,不可不加谨也。” \\
李若金曰:“贞者识字愈贞,淫者不识字亦淫。”
\end{comments}
\end{yulu}

\begin{yulu}{善读书者与善游山水者}
善读书者,无之而非书:山水亦书也,棋酒亦书也,花月亦书也;善游山水者,无之而非山水:书史亦山水也,诗酒亦山水也,花月亦山水也。
\begin{comments}
陈鹤山曰:“此方是真善读书人,善游山水人。” \\
黄交三曰:“善于领会者,当作如是观。” \\
江含徵曰:“五更卧被,时有无数山水书籍在眼前胸中。” \\
尤悔庵曰:“山耶,水耶,书耶,一而二,二而三,三而一者也。” \\
陆云士曰:“妙舌如环,真慧业文人之语。”
\end{comments}
\end{yulu}

\begin{yulu}{园亭之妙在丘壑布置}
园亭之妙,在丘壑布置,不在雕绘琐屑。往往见人家园亭,屋脊墙头,雕砖镂瓦,非不穷极工巧,然未久即坏,坏后极难修葺,是何如朴素之为佳乎。
\begin{comments}
江含徵曰:“世间最令人神怆者,莫如名园雅墅,一经颓废,风台月榭,埋没荆棘。故昔之贤达,有不欲置别业者。予尝过琴虞,留题名园,句有云:‘而今绮砌雕栏在,賸与园丁作业钱。’盖伤之也。” \\
弟木山曰:“予尝悟作园亭与作光棍二法。园亭之善,在多回廊;光棍之恶,在能结讼。”
\end{comments}
\end{yulu}

\begin{yulu}{清宵邀月,良夜呼蛩}
清宵独坐,邀月言愁;良夜孤眠,呼蛩语恨。
\begin{comments}
袁士旦曰:“令我百端交集。” \\
黄孔植曰:“此逆旅无聊之况,心斋亦知之乎!”
\end{comments}
\end{yulu}

\begin{yulu}{官声采于舆论,花案定于成心}
官声采于舆论,豪右之口与寒乞之口,俱不得其真;花案定于成心,艳媚之评与寝陋之评,概恐失其实。
\begin{comments}
黄九烟曰:“先师有言:‘不如乡人之善者,好之;其不善者,恶之。’” \\
李若金曰:“豪右而不讲分上,寒乞而不望推恩者,亦未尝无公论。” \\
倪永清曰:“我谓众人唾骂者,其人必有可观。”
\end{comments}
\end{yulu}

\begin{yulu}{胸藏丘壑,兴寄烟霞}
胸藏丘壑,城市不异山林;兴寄烟霞,阎浮有如蓬岛。
\begin{comments}

\end{comments}
\end{yulu}

\begin{yulu}{俗言之不足据}
梧桐为植物中清品,而形家独忌之,甚且谓“梧桐大如斗,主人往外走。”若竟视为不祥之物也者。夫翦桐封弟,其为宫中之桐可知。而卜世最久者,莫过于周,俗言之不足据,类如此夫!
\begin{comments}
江含征曰:“爱碧梧者,遂艰于白镪。造物盖忌之,故靳之也。有何吉凶、休咎之可关!衹是打秋风时,光棍样可厌耳!” \\
尤悔庵曰:“‘梧桐生矣,于彼朝阳’,《诗》言之矣。” \\
倪永清曰:“心斋为梧桐雪千古之奇冤,百卉俱当九顿。”
\end{comments}
\end{yulu}

\begin{yulu}{多情者、好饮者、喜读书者}
多情者不以生死易心,好饮者不以寒暑改量,喜读书者不以忙闲作辍。
\begin{comments}
朱其恭曰:“此三言者,皆是心斋自为写照。” \\
王司直日:“我愿饮酒、读《离骚》,至死方辍,何如?”
\end{comments}
\end{yulu}

\begin{yulu}{蛛与驴}
蛛为蝶之敌国,驴为马之附庸。
\begin{comments}
周星远曰:“妙论解颐,不数晋人危语、隐语。” \\
黄交三曰:“自开闢以来,未闻有此奇论。”
\end{comments}
\end{yulu}

\begin{yulu}{立品与渉世}
立品须发乎宋人之道学,渉世须参以晋代之风流。
\begin{comments}
方宝臣曰:“真道学,未有不风流者。” \\
张竹坡曰:“夫子自道也。” \\
胡静夫曰:“予赠金陵前辈赵容庵句云:‘文章鼎立《庄》《骚》外,杖履风流晋宋间。’今当移赠山老。” \\
倪永清曰:“等闲地位,却是个双料圣人。” \\
陆云士曰:“有不风流之道学,有风流之道学;有不道学风流,有道学之风流,毫厘千里。”
\end{comments}
\end{yulu}

\begin{yulu}{草木亦知人伦}
古谓禽兽亦知人伦。予谓匪独禽兽也,即草木亦复有之。牡丹为王,芍药为相,其君臣也;南山之乔,北山之梓,其父子也;荆之闻分而枯,闻不分而活,其兄弟也;莲之并蒂,其夫妇也;兰之同心,其朋友也。
\begin{comments}
江含徵曰:“纲常伦理,今日几于扫地,合向花木鸟兽中求之。”又曰:“心斋不喜迂腐,此却有腐气。”
\end{comments}
\end{yulu}

\begin{yulu}{豪杰与文人}
豪杰易于圣贤,文人多于才子。
\begin{comments}
张竹坡曰:“豪杰不能为圣贤,圣贤未有不豪杰。文人才子亦然。”
\end{comments}
\end{yulu}

\begin{yulu}{隐仕与仙凡}
牛与马,一仕而一隐也;鹿与豕,一仙而一凡也。
\begin{comments}
杜茶村曰:“田单之火牛,亦曾效力疆场;至马之隐者,则绝无之矣。若武王归马于华山之阳,所谓‘勒令致仕’者也。” \\
张竹坡曰:“‘莫与儿孙作马牛’,盖为后人审出处语也。”
\end{comments}
\end{yulu}

\begin{yulu}{至文皆血涙所成}
古今至文,皆血泪所成。
\begin{comments}
吴晴岩曰:“山老《清泪痕》一书,细看皆是血泪。” \\
江含徵曰:“古今恶文,亦纯是血。”
\end{comments}
\end{yulu}

\begin{yulu}{“情”与“才”}
“情”之一字,所以维持世界;“才”之一字,所以粉饰乾坤。
\begin{comments}
吴雨若曰:“世界原从‘情’字生出,有夫妇,然后有父子;有父子,然后有兄弟;有兄弟,然后有朋友;有朋友,然后有君臣。” \\
释中洲曰:“‘情’与‘才’缺一不可。”
\end{comments}
\end{yulu}

\begin{yulu}{孔子与释迦}
孔子生于东鲁;东者生方,故礼乐文章,其道皆自无而有。释迦生于西方;西者死地,故受想行识,其教皆自有而无。
\begin{comments}
吴街南曰:“佛游东土,佛入生方;人望西天,岂知是寻死地。呜呼,西方之人兮,之死靡他!” \\
殷日戒曰:“孔子衹勉人生时用功,佛氏衹教人死时作主,各自一意。” \\
倪永清曰:“盘古生于天心,故其人在不有不无之间。”
\end{comments}
\end{yulu}

\begin{yulu}{水藉色于山,诗乞灵于酒}
有青山方有绿水,水惟借色于山;有美酒便有佳诗,诗亦乞灵于酒。
\begin{comments}
李圣许曰:“有青山绿水,乃可酌美酒而咏佳诗。是诗酒又发端于山水也。”
\end{comments}
\end{yulu}

\begin{yulu}{三讲学者}
严君平以卜讲学者也,孙思邈以医讲学者也,诸葛武侯以出师讲学者也。
\begin{comments}
殷日戒曰:“心斋殆又以《幽梦影》讲学者耶!” \\
戴田友曰:“如此讲学,才可称道学先生。”
\end{comments}
\end{yulu}

\begin{yulu}{雄雌有分}
人谓女美于男,禽则雄华于雌,兽则牝牡无分者也。
\begin{comments}
杜于皇曰:“人亦有男美于女者,此尚非确论。” \\
徐松之曰:“此是茶村兴到之言,亦非定论。”
\end{comments}
\end{yulu}

\begin{yulu}{三“无奈事”}
镜不幸而遇嫫母,砚不幸而遇俗子,剑不幸而遇庸将;皆无可奈何之事。
\begin{comments}
杨圣藻曰:“凡不幸者,皆可以此槪之。” \\
闵宾连日:“心斋案头无一佳砚,然诗、文绝无一点尘俗气。此又砚之大幸也。” \\
曹沖谷曰:“最无可奈何者,佳人定随痴汉。”
\end{comments}
\end{yulu}

\begin{yulu}{五“必当”事}
天下无书则已;有则必当读;无酒则已,有则必当饮;无名山则已,有则必当游;无花月则已,有则必当赏玩;无才子佳人则已,有则必当爱慕怜惜。
\begin{comments}
弟木山曰:“谈何容易!即吾家黄山,几能得一到耶?”
\end{comments}
\end{yulu}

\begin{yulu}{搦管拈毫当不负秋虫春鸟}
秋虫春鸟,尚能调声弄舌,时吐好音。我辈搦管拈毫,岂可甘作鸦鸣牛喘!
\begin{comments}

\end{comments}
\end{yulu}

\begin{yulu}{媸颜不与镜为仇}
媸颜陋质,不与镜为仇者,亦以镜为无知之死物耳。使镜而有知,必遭扑破矣。
\begin{comments}
江含徵曰:“镜而有知,遇若辈早已回避矣。” \\
张竹坡曰:“镜而有知,必当化媸为妍。”
\end{comments}
\end{yulu}

\begin{yulu}{百忍同居}
吾家公艺,恃百忍以同居。千古传为美谈。殊不知忍而至于百,则其家庭乖戾暌隔之处,正未易更仆数也。
\begin{comments}
江含征曰:“然除了一忍,更无别法。” \\
顾天石曰:“心斋此论,先得我心。忍以治家可耳;奈何进之高宗,使忍以养成武氏之祸哉!” \\
倪永清曰:“若用‘忍’字,则百犹嫌少。否则以‘剑’字处之,足矣。或曰‘出家’二字,足以处之。” \\
王安节曰:“惟其乖戾睽隔,是以要忍。”
\end{comments}
\end{yulu}

\begin{yulu}{九世同居不可以为法}
九世同居诚为盛事,然止当与割股庐墓者作一例看。可以为难矣,不可以为法也。
\begin{comments}
洪去芜曰:“古人原有‘父子异宫’之说。” \\
沈契掌曰:“必居天下之广居而后可。”
\end{comments}
\end{yulu}

\begin{yulu}{作文之法}
作文之法:意之曲折者,宜写之以显浅之词;理之显浅者,宜运之以曲折之笔;题之熟者,参之以新奇之想;题之庸者,深之以关系之论。至于窘者舒之使长,缛者删之使简,俚者文之使雅,闹者摄之使静,皆所谓裁制也。
\begin{comments}
陈康畴曰:“深得作文三昧语。” \\
张竹坡曰:“所谓节制之师。” \\
王丹麓曰:“文家秘旨,和盘托出,有功作者不浅。”
\end{comments}
\end{yulu}

\begin{yulu}{物中尤物}
笋为蔬中尤物;荔枝为果中尤物;蟹为水族中尤物;酒为饮食中尤物;月为天文中尤物;西湖为山水中尤物;词曲为文字中尤物。
\begin{comments}
张南村曰:“《幽梦影》可为书中尤物。” \\
陈鹤山曰:“此一则又为《幽梦影》中尤物。”
\end{comments}
\end{yulu}

\begin{yulu}{爱怜解语花}
买得一本好花,犹且爱护而怜惜之,矧其为解语花乎?
\begin{comments}
周星远曰:“性至之语,自是君身有仙骨,世人那得知其故耶!” \\
石天外曰:“此一副心,令我念佛数声。” \\
李若金曰:“花能解语,而落于粗恶武夫,或遭狮吼戕贼,虽欲爱护,何可得?” \\
王司直曰:“此言是恻隐之心,即是是非之心。”
\end{comments}
\end{yulu}

\begin{yulu}{观便面知人}
观手中便面,足以知其人之雅俗,足以识其人之交游。
\begin{comments}
李圣许曰:“今人以笔资匄名人书画,名人何尝与之交游!吾知其手中便面虽雅,而其人则俗甚也。心斋此条,犹非定论。” \\
毕蝎谷曰:“人苟肯以笔资匄名人书画,则其人犹有雅道存焉,世固有幷不爱此道者。” \\
钱目天曰:“二说皆然。”
\end{comments}
\end{yulu}

\begin{yulu}{水火皆可变不洁为至洁}
水为至污之所会归,火为至污之所不到。若变不洁为至洁,则水火皆然。
\begin{comments}
江含徵曰:“世间之物,宜投诸水火者不少,盖喜其变也。”
\end{comments}
\end{yulu}

\begin{yulu}{貌有丑而可观者,文有不通而可爱者}
貌有丑而可观者,有虽不丑而不足观者;文有不通而可爱者,有虽通而极可厌者。此未易与浅人道也。
\begin{comments}
陈康畴曰:“相马于牝牡骊黄之外者,得之矣。” \\
李若金曰:“究竟可观者,必有奇怪处;可爱者,必无大不通。” \\
梅雪坪曰:“虽通而可厌,便可谓之不通。”
\end{comments}
\end{yulu}

\begin{yulu}{游玩山水须有缘}
游玩山水,亦复有缘。苟机缘未至,则虽近在数十里之内,亦无暇到也。
\begin{comments}
张南村曰:“予晤心斋时,询其曾游黄山否。心斋对以‘未游’,当是机缘未至耳。” \\
陆云士曰:“余慕心斋者十年。今戊寅之冬,始得一面。身到黄山恨其晚,而正未晚也。”
\end{comments}
\end{yulu}

\begin{yulu}{贫富戒谄骄}
贫而无谄,富而无骄,古人之所贤也;贫而无骄,富而无谄,今人之所少也。足以知世风之降矣。
\begin{comments}
许耒庵曰:“战国时已有‘贫贱骄人’之说矣。” \\
张竹坡曰:“有一人一时而对此谄、对彼骄者,更难!”
\end{comments}
\end{yulu}

\begin{yulu}{“十年读书,十年游山”恐亦不足偿愿}
昔人欲以十年读书,十年游山,十年检藏。予谓检藏尽可不必十年,只二三载足矣。若读书与游山,虽或相倍蓰,恐亦不足以偿所愿也。必也,如黄九烟前辈之所云:“人生必三百岁而后可乎!”
\begin{comments}
江含徵曰:“昔贤原谓尽则安能,但身到处,莫放过耳。” \\
孙松坪曰:“吾乡李长蘅先生爱湖上诸山,有‘每个峰头住一年’之句。然则黄九烟先生所云,犹恨其少。” \\
张竹坡曰:“今日想来,彭祖反不如马迁。”
\end{comments}
\end{yulu}

\begin{yulu}{毋为君子所鄙,毋为名宿所不知}
宁为小人之所骂,毋为君子之所鄙;宁为盲主司之所摈弃,毋为诸名宿之所不知。
\begin{comments}
陈康畴曰:“世之人自今以后,慎毋骂心斋也。” \\
江含征曰:“不独骂也,即打亦无妨,但恐鸡肋不足以安尊拳耳!” \\
张竹坡曰:“后二句足少平吾恨。” \\
李若金曰:“不为小人所骂,便是乡愿;若为君子所鄙,断非佳士。”
\end{comments}
\end{yulu}

\begin{yulu}{傲骨与傲心}
傲骨不可无,傲心不可有。无傲骨则近于鄙夫,有傲心不得为君子。
\begin{comments}
吴街南曰:“立君子之侧,骨亦不可傲;当鄙夫之前,心亦不可不傲。” \\
石天外曰:“道学之言,才人之笔。” \\
庞笔奴曰:“现身说法,真实妙谛。”
\end{comments}
\end{yulu}

\begin{yulu}{蝉与蜂}
蝉为虫中之夷、齐,蜂为虫中之管、晏。
\begin{comments}
崔青峙曰:“心斋可谓虫之董狐。” \\
吴镜秋曰:“蚊是虫中酷吏,蝇是虫中游客。”
\end{comments}
\end{yulu}

\begin{yulu}{人乐居“痴”、“愚”、“拙”、“狂”}
曰“痴”、曰“愚”、曰“拙”、曰“狂”,皆非好字面,而人每乐居之;曰“姦”、曰“黠”、曰“强”、曰“佞”,反是,而人每不乐居之,何也?
\begin{comments}
江含徵曰:“有其名者无其实,有其实者避其名。”
\end{comments}
\end{yulu}

\begin{yulu}{唐虞之乐可感鸟兽}
唐虞之际,音乐可感鸟兽。此盖唐虞之鸟兽,故可感耳。若后世之鸟兽,恐未必然。
\begin{comments}
洪去芜曰:“然则鸟兽亦随世道为升降耶?” \\
陈康畴曰:“后世之鸟兽,应是后世之人所化身,即不无升降,正未可知。” \\
石天外曰:“鸟兽自是可感,但无唐虞音乐耳。” \\
毕右万曰:“后世之鸟兽,与唐虞无异,但后世之人迥不同耳!”
\end{comments}
\end{yulu}

\begin{yulu}{痛痒与苦酸}
痛可忍,而痒不可忍;苦可耐,而酸不可耐。
\begin{comments}
陈康畴曰:“余见酸子偏不耐苦。” \\
张竹坡曰:“是痛、痒关心语。” \\
余香祖曰:“痒不可忍,须倩麻姑搔背。” \\
释牧堂曰:“若知痛痒、辨苦酸,便是居士悟处。”
\end{comments}
\end{yulu}

\begin{yulu}{镜中影与月下影}
镜中之影,著色人物也;月下之影,写意人物也。镜中之影,钩边画也;月下之影,没骨画也。月中山河之影,天文中地理也;水中星月之象,地理中天文也。
\begin{comments}
恽叔子曰:“绘空镂影之笔。” \\
石天外曰:“此种著色写意,能令古今善画人一齐搁笔。” \\
沈契掌曰:“好影子俱被心斋先生画著。”
\end{comments}
\end{yulu}

\begin{yulu}{读无字书与会难通解}
能读无字之书,方可得惊人妙句;能会难通之解,方可参最上禅机。
\begin{comments}
黄交三曰:“山老之学,从悟而入,故常有彻天彻地之言。”
\end{comments}
\end{yulu}

\begin{yulu}{诗酒与佳丽}
若无诗酒,则山水为具文;若无佳丽,则花月皆虚设。
\begin{comments}

\end{comments}
\end{yulu}

\begin{yulu}{造物之所忌}
才子而美姿容,佳人而工著作,断不能永年者,匪独为造物之所忌。盖此种原不独为一时之宝,乃古今万世之宝,故不欲久留人世以取亵耳!
\begin{comments}
郑破水曰:“千古伤心,同声一哭。” \\
王司直曰:“千古伤心者,读此可以不哭矣!”
\end{comments}
\end{yulu}

\begin{yulu}{北音}
陈平封曲逆侯,《史》、《汉》注皆云:“音去遇。”予谓此是北人土音耳。若南人四音俱全,似仍当读作本音为是。〈北人于唱“曲”之“曲”,亦读如“去”字。〉
\begin{comments}
孙松坪曰:“曲逆,今定县也。众水潆洄,势曲而流逆。予尝为土人订之,心斋重发吾覆矣。”
\end{comments}
\end{yulu}

\begin{yulu}{《诗经》时似有入声}
古人四声俱备,如“六”、“国”二字,皆入声也。今梨园演苏秦剧,必读“六”为“溜”,读“国”为“鬼”,从无读入声者。然考之《诗经》,如“良马六之、无衣六兮”之类,皆不与去声叶,而叶“祝”“告”“燠”;“国”字皆不与上声叶,而叶入“陌”“质”韵,则是古人似亦有入声,未必尽读“六”为“溜”、读“国”为“鬼”也。
\begin{comments}
弟木山曰:“梨园演苏秦,原不尽读‘六国’为‘溜鬼’,大抵以曲调为别。若曲是南调,则仍读入声也。”
\end{comments}
\end{yulu}

\begin{yulu}{砚佳与妾美}
闲人之砚,固欲其佳;而忙人之砚,尤不可不佳。娱情之妾,固欲其美;而广嗣之妾,亦不可不美。
\begin{comments}
江含徵曰:“砚美下墨可也,妾美招妒奈何?” \\
张竹坡曰:“妒在妾,不在美。”
\end{comments}
\end{yulu}

\begin{yulu}{独乐乐、与人乐乐、与众乐乐}
如何是独乐乐,曰鼓琴。如何是与人乐乐,曰奕棋。如何是与众乐乐,曰马吊。
\begin{comments}
蔡铉昇曰:“独乐乐,与人乐乐,孰乐?曰‘不若与人’。与少乐乐,与众乐乐,孰乐?曰‘不若与少’。” \\
王丹麓曰:“我与蔡君异。独畏人为鬼阵,见则必乱其局而后已。”
\end{comments}
\end{yulu}

\begin{yulu}{善恶与胎卵湿化四生}
不待教而为善为恶者,胎生也。必待教而后为善为恶者,卵生也。偶因一事之感触,而突然为善为恶者,湿生也。〈如周处、戴渊之改过,李怀光反叛之类。〉前后判若两截,究非一日之故也,化生也。〈如唐玄宗、卫武公之类。〉
\begin{comments}

\end{comments}
\end{yulu}


\begin{yulu}{物以神用者}
凡物皆以形用。其以神用者,则镜也、符印也、日晷也、指南针也。
\begin{comments}
袁中江曰:“凡人皆以形用。其以神用者,圣贤也,仙也,佛也。” \\
黄虞外士曰:“凡物之用皆形,而其所以然者,神也。镜凸凹而易其肥瘦,符印以专一而主其神机,日晷以恰当而定准则,指南以灵动而活其针缝。是皆神而明之存乎人矣。”
\end{comments}
\end{yulu}

\begin{yulu}{愿来世托生为绝代佳人}
才子遇才子,每有怜才之心;美人遇美人,必无惜美之意。我愿来世托生为绝代佳人,一反其局而后快。
\begin{comments}
陈鹤山曰:“谚云:‘鲍老当筵笑郭郎,笑他舞袖大郎当。若教鲍老当筵舞,转更郎当舞袖长。’则为之奈何?” \\
郑蕃修曰:“俟心斋来世为佳人时再议。” \\
余湘客曰:“古亦有‘我见犹怜’者。” \\
倪永清曰:“再来时不可忘却。”
\end{comments}
\end{yulu}

\begin{yulu}{建无遮大会}
予尝欲建一无遮大会,一祭历代才子,一祭历代佳人。俟遇有真正高僧,即当为之。
\begin{comments}
顾天石曰:“君若果有此盛举,请迟至二三十年之后,则我亦可以拜领盛情也。” \\
释中洲曰:“我是真正高僧,请即为之,何如?不然,则此二种沉魂滞魄,何日而得解脱耶?” \\
江含徵曰:“折柬虽具,而未有定期,则才子佳人亦复怨声载道。”又曰:“我恐非才子而冒为才子,非佳人而冒为佳人。虽有十万八千母陀罗臂,亦不能具香厨法膳也,心斋以为然否?” \\
释远峰曰:“中洲和尚,不得夺我施主!”
\end{comments}
\end{yulu}

\begin{yulu}{天地之替身}
圣贤者,天地之替身。
\begin{comments}
石天外曰:“此语大有功名教,敢不伏地拜倒!” \\
张竹坡曰:“圣贤者,乾坤之帮手。”
\end{comments}
\end{yulu}

\begin{yulu}{天极不难做}
天极不难做,只须生仁人君子有才德者二三十人足矣。君一、相一、冢宰一,及诸路总制抚军是也。
\begin{comments}
黄九烟曰:“吴歌有云:‘做天切莫做四月天。’可见天亦有难做之时。” \\
江含征曰:“天若好做,又不须女娲氏补之。” \\
尤谨庸曰:“天不做天,衹是做梦。奈何!奈何!” \\
倪永清曰:“天若都生善人,君相皆当袖手,便可无为而治。” \\
陆云士曰:“极诞极奇之话,极真极确之话。”
\end{comments}
\end{yulu}

\begin{yulu}{一登仕版辄与掷“陞官图”反}
掷“陞官图”,所重在“德”,所忌在“赃”。何一登仕版,辄与之相反耶?
\begin{comments}
江含征曰:“所重在‘德’,不过是要赢几文钱耳!” \\
沈契掌曰:“仕版原与纸版不同!”
\end{comments}
\end{yulu}

\begin{yulu}{动植之三教}
动物中有三教焉:蛟龙麟凤之属,近于儒者也;猿狐鹤鹿之属,近于仙者也;狮子牯牛之属,近于释者也。植物中有三教焉;竹梧兰蕙之属,近于儒者也;蟠桃老桂之属,近于仙者也;莲花薝卜之属,近于释者也。
\begin{comments}
顾天石曰:“请高唱《西厢》一句,一个通彻三教九流。” \\
石天外曰:“众人碌碌,动物中蜉蝣而已;世人峥嵘,植物中荆棘而已。”
\end{comments}
\end{yulu}

\begin{yulu}{佛言喩人身}
佛氏云:“日月在须弥山腰。”果尔,则日月必是绕山横行而后可。苟有昇有降,必为山巓所碍矣。又云:“地上有阿耨达池,其水四出,流入诸印度。”又云:“地轮之下为水轮,水轮之下为风轮,风轮之下为空轮。”余谓此皆喩言人身也:须弥山喩人首,日月喩两目,池水四出喩血脉流通,地轮喩此身,水为便溺,风为泄气。此下则无物矣。
\begin{comments}
释远峰曰:“却被此公道破。” \\
毕右万曰:“乾坤交后,有三股大气:一呼吸、二盘旋、三升降。呼吸之气,在八卦为震、巽,在天地为风、雷,为海潮,在人身为鼻息;盘旋之气,在八卦为坎、离,在天地为日、月,在人身为两目,为指尖、发顶罗纹,在草木为树节、蕉心;升降之气,在八卦为艮、兑,在天地为山、泽,在人身为髓液便溺,为头颅、肚腹,在草木为花叶之萌、雕,为树梢之向天、树根之入地。知此,而寓言之出于二氏者,皆可类推而悟。”
\end{comments}
\end{yulu}

\begin{yulu}{补东坡和陶诗}
苏东坡和陶诗,尚遗数十首。予尝欲集坡句以补之,苦于韵之弗备而止。如《责子诗》中“不识六与七”,“但觅梨与栗”,“七”字、“栗”字皆无其韵也。
\begin{comments}

\end{comments}
\end{yulu}

\begin{yulu}{偶得之句}
予尝偶得句,亦殊可喜,惜无佳对,遂未成诗。其一为“枯叶带虫飞”,其一为“乡月大于城”。姑存之,以俟异日。
\begin{comments}

\end{comments}
\end{yulu}

\begin{yulu}{四妙境}
“空山无人,水流花开”二句,极琴心之妙境。“胜固欣然,败亦可喜”二句,极手谈之妙境。“帆随湘转,望衡九面”二句,极泛舟之妙境。“胡然而天,胡然而帝”二句,极美人之妙境。
\begin{comments}

\end{comments}
\end{yulu}

\begin{yulu}{影之“受”“施”}
镜与水之影,所受者也;日与灯之影,所施者也。月之有影,则在天者为受,而在地者为施也。
\begin{comments}
郑破水曰:“‘受’、‘施’二字,深得阴阳之理。” \\
庞天池曰:“幽梦之影,在心斋为施,在笔奴为受。”
\end{comments}
\end{yulu}

\begin{yulu}{水声、风声、雨声}
水之为声有四:有瀑布声,有流泉声,有滩声,有沟浍声。风之为声有三:有松涛声,有秋叶声,有波浪声;雨之为声有二:有梧蕉荷叶上声,有承檐溜竹筒中声。
\begin{comments}
弟木山曰:“数声之中,惟水声最为厌。以其无已时,甚聒人耳也。”
\end{comments}
\end{yulu}

\begin{yulu}{文人好鄙富人}
文人每好鄙薄富人,然于诗文之佳者,又往往以金玉珠玑锦绣誉之,则又何也?
\begin{comments}
陈鹤山曰:“犹之富贵家张山癯野老落木荒村之画耳。” \\
江含徵曰:“富人嫌其悭且俗耳,非嫌其珠玉文绣也。” \\
张竹坡曰:“不文,虽穷可鄙;能文,虽富可敬。” \\
陆云士曰:“竹坡之言,是真公道说话!” \\
李若金曰:“富人之可鄙者,在吝,或不好史、书,或畏交游,或趋炎热而轻忽寒士。若非然者,则富翁大有裨益人处,何可少之!”
\end{comments}
\end{yulu}

\begin{yulu}{闲与忙}
能闲世人之所忙者,方能忙世人之所闲。
\begin{comments}

\end{comments}
\end{yulu}

\begin{yulu}{读经史}
先读经,后读史,则论事不谬于圣贤。既读史,复读经,则观书不徒为章句。
\begin{comments}
黄交三曰:“宋儒语录中不可多得之句。” \\
陆云士曰:“先儒著书法,累牍连章,不若心斋数言道尽。” \\
王宓草曰:“妄论经、史者,还宜退而读经。”
\end{comments}
\end{yulu}

\begin{yulu}{城市之乐}
居城市中,当以画幅当山水,以盆景当苑囿,以书籍当友朋。
\begin{comments}
周星远曰:“究是心斋偏重独乐乐!” \\
王司直曰:“心斋先生置身于画中矣!”
\end{comments}
\end{yulu}

\begin{yulu}{乡居须得良朋}
乡居须得良朋始佳。若田夫樵子,仅能辨五谷而测晴雨,久且数,未免生厌矣。而友之中,又当以能诗为第一,能谈次之,能画次之,能歌又次之,解觞政者又次之。
\begin{comments}
江含征曰:“说鬼话者,又次之。” \\
殷日戒曰:“奔走于富贵之门者,自应以善说鬼话为第一,而诸客次之。” \\
倪永清曰:“能诗者,必能说鬼话。” \\
陆云士曰:“三说递进,愈转愈妙,滑稽之雄。”
\end{comments}
\end{yulu}

\begin{yulu}{花鸟中之夷尹惠}
玉兰,花中之伯夷也〈高而且洁〉。葵,花中之伊尹也〈倾心向日〉。莲,花中之柳下惠也〈污泥不染〉。鹤,鸟中之伯夷也〈仙品也〉。鸡,鸟中之伊尹也〈司晨〉。莺,鸟中之柳下惠也〈求友〉。
\begin{comments}

\end{comments}
\end{yulu}

\begin{yulu}{名不符实}
无其罪而虚受恶名者,蠹鱼也〈蛀书之虫,另是一种,其形如蚕蛹而差小〉;有其罪而恒逃清议者,蜘蛛也。
\begin{comments}
张竹坡曰:“自是老吏断狱。” \\
李若金曰:“予尝有除蛛网说,则讨之未尝无人。”
\end{comments}
\end{yulu}

\begin{yulu}{臭腐化神奇}
臭腐化为神奇,酱也、腐乳也、金汁也。至神奇化为臭腐,则是物皆然。
\begin{comments}
袁中江曰:“神奇不化臭腐者,黄金也、真诗文也。” \\
王司直曰:“曹操、王安石文字,亦是神奇出于臭腐。”
\end{comments}
\end{yulu}

\begin{yulu}{君子、小人相攻之大势}
黑与白交,黑能污白,白不能掩黑;香与臭混,臭能胜香,香不能敌臭。此君子、小人相攻之大势也。
\begin{comments}
弟木山曰:“人必喜白而恶黑,黜臭而取香,此又君子必胜小人之理也。理在,又乌论乎势!” \\
石天外曰:“余尝言:于黑处著一些白,人必惊心骇目,皆知黑处有白;于白处著一些黑,人亦必惊心骇目,以为白处有黑。甚矣!君子之易于形短,小人之易于见长,此不虞之誉、求全之毁所由来也。读此慨然。” \\
倪永清曰:“当今以臭攻臭者不少。”
\end{comments}
\end{yulu}

\begin{yulu}{“耻”“痛”二字}
“耻”之一字,所以治君子;“痛”之一字,所以治小人。
\begin{comments}
张竹坡曰:“若使君子以耻治小人,则有耻且格;小人以痛报君子,则尽忠报国。”
\end{comments}
\end{yulu}

\begin{yulu}{当局则迷}
镜不能自照,衡不能自权,剑不能自击。
\begin{comments}
倪永清曰:“诗不能自传,文不能自誉。” \\
庞天池曰:“美不能自见,恶不能自掩。”
\end{comments}
\end{yulu}

\begin{yulu}{诗亦不必待穷而后工}
古人云:“诗必穷而后工”。盖穷则语多感慨,易于见长耳。若富贵中人,既不可忧贫叹贱,所谈者不过风云月露而已,诗安得佳?苟思所变,计惟有出游一法,即以所见之山川风土物产人情,或当疮痍兵燹之馀,或值旱潦灾祲之后,无一不可寓之诗中,藉他人之穷愁,以供我之咏叹,则诗亦不必待穷而后工也。
\begin{comments}
张竹坡曰:“所以郑监门《流民图》独步千古。” \\
倪永清曰:“得意之游,不暇作诗;失意之游,不能作诗。苟能以无意游之,则眼光识力定是不同。” \\
尤悔庵曰:“世之穷者多而工诗者少,诗亦不任受过也。”
\end{comments}
\end{yulu}
