\documentclass{book}


%========基本必备的宏包========%

\usepackage{geometry}
\usepackage{ctex}
\usepackage{setspace} % 加载 setspace 宏包,用于设置行间距


% 定义 comment 环境
\newenvironment{comments}
{%
    \hangindent=1em % 设置缩进
    \leftskip = 1em % 左侧缩进
    \par\vspace{0.3cm} % 段落换行并添加垂直间距
    \fontsize{11}{12}
}{%
    \par % 段落换行
}

% 定义 yulu 环境
\newenvironment{yulu}
{%
    \setlength{\parindent}{0pt} % 在 yulu 环境中取消首行缩进
    \setstretch{1.2} % 设置行间距为 1.2 倍
    \ignorespaces % 忽略环境开始处的空格
    \fontsize{12}{14}
}{%
    \par\bigskip % 环境结束后添加较大的垂直间距
}


\title{幽梦影}
\author{张潮}
\date{\today}

\begin{document}

\maketitle

\chapter*{前言}
《幽梦影》是清代文学家张潮所著的一部随笔集。书中内容涉及广泛,涵盖了文学、艺术、人生哲理等多个方面。

\chapter{正文}

\begin{yulu}

读经宜冬,其神专也;读史宜夏,其时久也;读诸子宜秋,其致别也;读诸集宜春,其机畅也。

\begin{comments}
曹秋岳曰:“可想见其南面百城时。”

庞笔奴曰:“读《幽梦影》则春、夏、秋、冬,无时不宜。”
\end{comments}
\end{yulu}

\begin{yulu}

经传宜独坐读;史鉴宜与友共读。

\begin{comments}
孙恺似曰:“深得此中真趣,固难为不知者道。”  

王景州曰:“如无好友,即红友亦可。”
\end{comments}

\end{yulu}

\end{document}
