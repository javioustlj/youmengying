\chapter*{序}

\section{余怀序}

余穷经读史之馀,好览稗官小说。自唐以来,不下数百种。不但可以备考遗忘,亦可以增长意识。如游名山大川者,必探断崖绝壑;玩乔松古柏者,必采秀草幽花。使耳目一新,襟情怡宕。此非头巾褦襶章句腐儒之所知也。故余于咏诗撰文之暇,笔录古轶事、今新闻。自少至老,杂著数十种。如《说史》、《说诗》、《党鉴》、《盈鉴》、《东山谈苑》、《汗青馀语》、《砚林》、《不妄语述》、《茶史补》、《四莲花斋杂录》、《曼翁漫录》、《禅林漫录》、《读史浮白集》、《古今书字辨讹》、《秋雪丛谈》、《金陵野抄》之类。虽未雕板问世,而友人借抄,几遍东南诸郡,直可傲子云而睨君山矣!天都张仲子心斋,家积缥缃,胸罗星宿,笔花缭绕,墨渖淋漓。其所著述,与余旗鼓相当,争奇斗富,如孙伯符与太史子义相遇于神亭,又如石崇、王恺击碎珊瑚时也。其《幽梦影》一书,尤多格言妙论,言人之所不能言,道人之所未经道。展味低徊,似餐帝浆沆瀣,听钧天之广乐,不知此身在下方尘世矣。至如:“律己宜带秋气,处世宜带春气”、“婢可以当奴,奴不可以当婢”、“无损于世谓之善人,有害于世谓之恶人”、“寻乐境乃学仙,避苦境乃学佛”,超超玄箸,绝胜支、许清谈。人当镂心铭腑,岂止佩韦书绅而已哉。



鬘持老人余怀广霞制


\section{孙致弥序}

心斋著书满家,皆含经咀史,自出机杼,卓然可传。是编是其一脔片羽,然三才之理,万物之情,古今人事之变,皆在是矣。顾题之以“梦”且“影”云者,吾闻海外有国焉,夜长而昼短,以昼之所为为幻,以梦之所遇为真;又闻人有恶其影而欲逃之者。然则“梦”也者,乃其所以为觉;“影”也者,乃其所以为形也耶?廋辞隐语,言无罪而闻足戒,是则心斋所为尽心焉者也。读是编也,其可以闻破梦之钟而就阴以息影也夫!



江东同学弟孙致弥题


\section{石庞序}

张心斋先生家自黄山,才奔陆海,柟榴赋就;锦月投怀,芍药词成,繁花作馔。苏子瞻十三楼外,景物犹然;杜枚之廿四桥头,流风仍在。静能见性,洵哉人我不间而喜嗔不形;弱仅胜衣,或者清虚日来而滓秽日去。怜才惜玉,心是灵犀;绣腹锦胸,身同丹凤。花间选句,尽来珠玉之音;月下题词,已满珊瑚之笥。岂如兰台作赋,仅别东西;漆园著书,徒分内外而已哉!然而繁文艳语,止才子馀能;而卓识奇思,诚词人本色。若夫舒性情而为著述,缘阅历以作篇章,清如梵室之钟,令人猛省;响若尼山之铎,别有深思。则《幽梦影》一书,余诚不能已于手舞足蹈、心旷神怡也。其云“益人谓善,害物谓恶”,咸仿佛乎外王内圣之言。又谓“律己宜秋,处世宜春”,亦陶溶乎诚意正心之旨。他如片花寸草均有会心;遥水近山不遗玄想;息机物外,古人之糟粕不论;信手拈时,造化之精微入悟;湖山乘兴,尽可投囊;风月维谭,兼供挥麈;金绳觉路,弘开入梦之毫;宝筏迷津,直渡广长之舌。以风流为道学,寓教化于诙谐。为色为空,知犹有这个在;如梦如影,且应做如是观。



湖上晦村学人石庞序


\section{松溪王序}

记曰:“和顺积于中,英华发于外。”凡文人之立言,皆英华之发于外者也。无不本乎中之积,而适与其人肖焉。是故其人贤者,其言雅;其人哲者,其言快;其人高者,其言爽;其人达者,其言旷;其人奇者,其言创;其人韵者,其言多情思。张子所云:“对渊博友如读异书,对风雅友如读名人诗文,对谨饬友如读圣贤经传,对滑稽友如阅传奇小说。正此意也。”彼在昔立言之人,到今传者,岂徒传其言哉!传其人而已矣。今举集中之言,有快若幷州之剪,有爽若哀家之梨,有雅若钧天之奏,有旷若空谷之音;创者则如新锦出机,多情则如游丝袅树。以为贤人可也,以为达人、奇人可也,以为哲人可也。譬之瀛洲之木,日中视之,一叶百形。张子以一人而兼众妙,其殆瀛木之影欤?然则阅乎此一编,不啻与张子晤对,罄彼我之怀!又奚俟梦中相寻,以致迷不知路,中道而返哉!

同学弟松溪王 拜题
